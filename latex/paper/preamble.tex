% Margins ----------------------------------------------------------------------
\usepackage[margin=1.25in]{geometry}


% Line Spacing -----------------------------------------------------------------
\renewcommand{\baselinestretch}{1.5}


% AMS --------------------------------------------------------------------------
\usepackage{amsmath}
\usepackage{amsfonts}
\usepackage{amsthm}
\usepackage{graphicx}


% Define Theorems --------------------------------------------------------------
% note, theorem is the name that goes in \begin{} and Theorem is the name displayed as Theorem 1

% Put proper spacing after Theorem #. 
\newtheoremstyle{spacing}
{}%          Space above, empty = `usual value'
{}%          Space below
{\itshape}%  Body font
{}%          Indent amount (empty = no indent, \parindent = para indent)
{\bfseries\color{navyblue}}% Thm head font
{.}%         Punctuation after thm head
{2.5mm}%  Space after thm head: \newline = linebreak
{}%          Thm head spec

\theoremstyle{spacing}
\newtheorem{theorem}{Theorem}
\newtheorem{proposition}{Proposition}
\newtheorem{assumption}{Assumption}
\newtheorem{example}{Example}


% Font -------------------------------------------------------------------------
\usepackage[T1]{fontenc}
\usepackage[utopia, varg]{newtxmath}
\renewcommand{\rmdefault}{futs} % Utopia as text font 

% Small adjustments to text kerning
\usepackage{microtype}


% Remove annoying over-full box warnings ---------------------------------------
\vfuzz2pt 
\hfuzz2pt


% Tikz support -----------------------------------------------------------------
\usepackage{tikz}


% Color Palette ----------------------------------------------------------------
\usepackage{xcolor}

% https://www.materialpalette.com/colors
\definecolor{red}{HTML}{c62828}
\definecolor{orange}{HTML}{ef6c00}
\definecolor{green}{HTML}{2e7d32}
\definecolor{blue}{HTML}{1565c0}
\definecolor{purple}{HTML}{283593}
\definecolor{maroon}{HTML}{AF3335}
\definecolor{dark-maroon}{HTML}{5D0F0D}
\definecolor{teal}{HTML}{00695c}
\definecolor{bluegrey}{HTML}{455a64}
\definecolor{indigo}{HTML}{1A237E}
\definecolor{navyblue}{HTML}{0A3044}
\definecolor{bluegreen}{HTML}{4A8676}
\definecolor{Black}{HTML}{000000}

% CU Boulder colors
\definecolor{buff-gold}{HTML}{CFB87C}
\definecolor{buff-grey}{HTML}{565A5C}
\definecolor{buff-lightgrey}{HTML}{A2A4A3}
\definecolor{buff-black}{HTML}{000000}


% Hyperlinks -------------------------------------------------------------------
\usepackage{hyperref}
\hypersetup{
    colorlinks= true,
    citecolor= dark-maroon,
    linkcolor= dark-maroon,
    filecolor= dark-maroon,      
    urlcolor= dark-maroon,
}


% Citations --------------------------------------------------------------------
%\usepackage[style= authoryear, natbib= true, backend= bibtex]{biblatex}
\usepackage{natbib}
\bibliographystyle{econ-aea}


% Enumerate/Itemize ------------------------------------------------------------
\usepackage{enumitem}
\setitemize{labelindent=0.5em,labelsep=0.25cm,leftmargin=*}
\setenumerate{labelindent=0.5em,labelsep=0.25cm,leftmargin=*}


% Section and Subsection Styling -----------------------------------------------
\usepackage[explicit]{titlesec}

\titleformat{\section}
  {\Large \bf \color{navyblue}}
  {\thesection \,---}
  {0.25em}
  {#1}
  
\titleformat{\subsection}
  {\fontsize{11}{10}\bf}
  {\thesubsection.}
  {1em}
  {#1}

\titleformat{\subsubsection}
  {\fontsize{11}{10}\it}
  {\thesubsubsection.}
  {1em}
  {#1}


% Footnote ---------------------------------------------------------------------
% Spacing between footnotes on same page
\addtolength{\footnotesep}{1mm}

% Space after footnote number
\let\oldfootnote\footnote
\renewcommand\footnote[1]{\oldfootnote{\ #1}}


% Better Abstract --------------------------------------------------------------
\renewenvironment{abstract}
{
  \centerline
  {\large \bfseries \scshape \color{navyblue} Abstract}
  \begin{quote}
}
{
  \end{quote}
}


% Custom Math Definitions ------------------------------------------------------
\global\long\def\expec#1{\mathbb{E}\left[#1\right]}%
\global\long\def\prob#1{\mathbb{P}\left[#1\right]}%
\global\long\def\var#1{\mathrm{Var}\left[#1\right]}%
\global\long\def\cov#1{\mathrm{Cov}\left[#1\right]}%
\global\long\def\one{\mathbf{1}}%


% Table and Figure labelling ---------------------------------------------------
\usepackage{caption}
% multifigure with \caption
% \begin{subfigure} \end{subfigure}
\usepackage[skip=-8pt]{subcaption}

\DeclareCaptionLabelSeparator{threedash}{\,---\,}
\DeclareCaptionFont{navyblue}{\color{navyblue}}
\captionsetup[table]{format=plain, labelsep=threedash, font={navyblue, bf}}
\captionsetup[figure]{format=plain, labelsep=threedash, font={navyblue, bf}}

% Left align captions
% \captionsetup[table]{labelfont=it, textfont={navyblue, bf}, labelsep=newline, justification=raggedright, singlelinecheck=off}
% \captionsetup[figure]{labelfont=it, textfont={navyblue, bf}, labelsep=newline, justification=raggedright, singlelinecheck=off}


% Tables -----------------------------------------------------------------------

% Make tables/figures wider than paragraph using:
% \begin{adjustbox}{width = 1.2\textwidth, center}
\usepackage{adjustbox}
\usepackage{array}

% Slighty more spacing between rows
\renewcommand\arraystretch{1.1}

% If tables are too narrow, fill columns using:
% \begin{tabularx}{\linewidth}{cols}
% col-types: X - center, L - left, R -right
% If you want relative scale for columns: 
% >{\hsize=.8\hsize}X/L/R

\usepackage{tabularx}
\newcolumntype{L}{>{\raggedright\arraybackslash}X}
\newcolumntype{R}{>{\raggedleft\arraybackslash}X}
\newcolumntype{C}{>{\centering\arraybackslash}X}

% Table with easy to use footnotes
% \begin{threeparttable}
%    \begin{tabular} ... \end{tabular}
%    \begin{tablenotes}
%        \item \textit{Notes.}
%    \end{tablenotes}  
% \end{threeparttable}
\usepackage[flushleft]{threeparttable}
\setlength\labelsep{0pt}

% \toprule, \cmidrule, \bottomrule
\usepackage{booktabs}

% Landscape table --------------------------------------------------------------
% \begin{landscape} \pagestyle{lscaped} table... \end{landscsape}
% \usepackage{pdflscape} - rotates page left-side up in pdf
% \usepackage{lscape} - does not rotate page, only figure/table

\usepackage{pdflscape}
% \usepackage{lscape}

% For landscape, fix page number location
\usepackage{fancyhdr}
\fancypagestyle{lscaped}{%
    \fancyhf{}
    \renewcommand{\headrulewidth}{0pt}
    \textnormal
    \fancyfoot{%
        \tikz[remember picture,overlay]
        \node[outer sep=2.5cm,above,rotate=90] at (current page.east) {\thepage};
    }
}
  

% ------------------------------------------------------------------------------