\documentclass[aspectratio=43]{beamer}
% Metropolis Theme ------------------------------------------------------------------------------
\usetheme{metropolis} % Use metropolis theme


% Title ------------------------------------------------------------------------------
\title{Differnces-in-Differences with Spatial Spillover}
\date{\today}
\author{Kyle Butts}
% \institue{}

% Margins ----------------------------------------------------------------------

\usepackage[margin=1.25in]{geometry}

% AMS --------------------------------------------------------------------------

\usepackage{amsmath}
\usepackage{amsfonts}
\usepackage{amsthm}
\usepackage{graphicx}


% Line Spacing -----------------------------------------------------------------

\renewcommand{\baselinestretch}{1.5}


% Font -------------------------------------------------------------------------

\usepackage[T1]{fontenc}
\usepackage[default]{lato} % Lato as text font
% \usepackage[utopia, varg]{newtxmath}
% \renewcommand{\rmdefault}{futs} % Utopia as text font 

% Small adjustments to text kerning
\usepackage{microtype}

% Remove annoying over-full box warnings
\vfuzz2pt 
\hfuzz2pt


% Tikz support -----------------------------------------------------------------

\usepackage{tikz}


% Color Palette ----------------------------------------------------------------

\usepackage{xcolor}

% https://www.materialpalette.com/colors
\definecolor{dark-maroon}{HTML}{5D0F0D}
\definecolor{navyblue}{HTML}{0A3044}

% From Davidson Mackinnon
\definecolor{dm-blue}{HTML}{086fbd}
\definecolor{dm-red}{HTML}{ba3132}
\definecolor{dm-green}{HTML}{3f7e32}

% https://www.viget.com/articles/color-contrast/
\definecolor{purple}{HTML}{5601A4}
\definecolor{navy}{HTML}{0D3D56}
\definecolor{ruby}{HTML}{9a2515}
\definecolor{alice}{HTML}{107895}
\definecolor{daisy}{HTML}{EBC944}
\definecolor{coral}{HTML}{F26D21}
\definecolor{kelly}{HTML}{829356}
\definecolor{cranberry}{HTML}{E64173}
\definecolor{jet}{HTML}{131516}
\definecolor{asher}{HTML}{555F61}
\definecolor{slate}{HTML}{314F4F}


% Hyperlinks -------------------------------------------------------------------

\usepackage{hyperref}
\hypersetup{
    colorlinks= true,
    citecolor= dark-maroon,
    linkcolor= dark-maroon,
    filecolor= dark-maroon,      
    urlcolor= dark-maroon,
}


% Citations --------------------------------------------------------------------

% note, natbib provides better hyperlinking
\usepackage{natbib}
\bibliographystyle{econ-aea}


% Define Theorems --------------------------------------------------------------

% Put proper spacing after Theorem #. 
\newtheoremstyle{spacing}
{}%          Space above, empty = `usual value'
{}%          Space below
{}%  Body font
{}%          Indent amount (empty = no indent, \parindent = para indent)
{\bfseries\color{navyblue}}% Thm head font
{.}%         Punctuation after thm head
{2.5mm}%  Space after thm head: \newline = linebreak
{}%          Thm head spec

% note, theorem is the name that goes in \begin{} and Theorem is the name displayed as Theorem 1
\theoremstyle{spacing}
\newtheorem{theorem}{Theorem}
\newtheorem{proposition}{Proposition}
\newtheorem{assumption}{Assumption}
\newtheorem{example}{Example}


% Custom Math Definitions ------------------------------------------------------

\newcommand{\expec}[1]{\mathbb{E}\left[#1\right]}%
\newcommand{\condexpec}[2]{\mathbb{E}\left[#1 \ \vert \ #2\right]}%
\newcommand{\prob}[1]{\mathbb{P}\left[#1\right]}%
\newcommand{\var}[1]{\mathrm{Var}\left[#1\right]}%
\newcommand{\cov}[1]{\mathrm{Cov}\left[#1\right]}%
\newcommand{\one}{\mathbf{1}}


% Titlepage --------------------------------------------------------------------

% \maketitle
\usepackage{titling}
\usepackage{setspace}

% title
\pretitle{\begin{spacing}{1}\begin{flushleft}\huge}
\posttitle{\end{flushleft}\end{spacing}\vspace{-5mm}}
% author, note don't use \and 
\preauthor{\begin{flushleft}\LARGE}
\postauthor{\end{flushleft}\vspace{-7.5mm}}
% date
\predate{\begin{flushleft}\Large\color{asher}}
\postdate{\end{flushleft}\vspace{-5mm}}

% Abstract
\renewenvironment{abstract}
 {\noindent\rule{\linewidth}{.5pt}\noindent}
 {\noindent\rule{\linewidth}{.5pt}}

% alternative abstract
% \renewenvironment{abstract}
% {
%   \centerline {\large \bfseries \scshape \color{navyblue} Abstract}
%   \begin{quote}
% }
% {\end{quote}}


% Section and Subsection Styling -----------------------------------------------

\usepackage[explicit]{titlesec}

\titleformat{\section}
  {\Large \bf \color{navyblue}}
  {\thesection \,---}
  {0.25em}
  {#1}
  
\titleformat{\subsection}
  {\fontsize{11}{10}\it}
  {\thesubsection.}
  {1em}
  {#1}

% Don't number subsubsection
\setcounter{secnumdepth}{2}

% Footnote ---------------------------------------------------------------------

% Spacing between footnotes on same page
\addtolength{\footnotesep}{1mm}

% Space after footnote number
\let\oldfootnote\footnote
\renewcommand\footnote[1]{\oldfootnote{\ #1}}

% No footnote line
\renewcommand\footnoterule{}

% No supsercript in footer
\makeatletter
\renewcommand\@makefntext[1]{%
    \parindent 1em \noindent
    \hb@xt@1.8em{\hss\normalfont\@thefnmark.\hfill}#1
  }
\makeatother




% Enumerate/Itemize ------------------------------------------------------------

\usepackage{enumitem}
\setitemize{labelindent=0.5em,labelsep=0.25cm,leftmargin=*}
\setenumerate{labelindent=0.5em,labelsep=0.25cm,leftmargin=*}


% Table and Figure labelling ---------------------------------------------------

\usepackage{caption}

\DeclareCaptionLabelSeparator{threedash}{\,---\,}
\DeclareCaptionFont{navyblue}{\color{navyblue}}
\DeclareCaptionFont{jet}{\color{jet}}
\captionsetup[table]{format=plain, labelsep=threedash, font={navyblue, bf}}
\captionsetup[figure]{format=plain, labelsep=threedash, font={navyblue, bf}}

% Alternative: Left align captions
% \captionsetup[table]{labelfont=it, textfont={navyblue, bf}, labelsep=newline, justification=raggedright, singlelinecheck=off}
% \captionsetup[figure]{labelfont=it, textfont={navyblue, bf}, labelsep=newline, justification=raggedright, singlelinecheck=off}

% multifigure with \caption
% \begin{subfigure}\caption{} \end{subfigure}
\usepackage{subcaption}
\captionsetup[subfigure]{format=plain, font={jet, footnotesize, bf}}


% Tables -----------------------------------------------------------------------

% Fix \input with tables
% \input fails when \\ is at end of external .tex file

\makeatletter
\let\input\@@input
\makeatother

% Make tables/figures wider than \textwidth using:
% \begin{adjustbox}{width = 1.2\textwidth, center}
% \end{adjustbox}
\usepackage{adjustbox}

% Slighty more spacing between rows
\usepackage{array}
\renewcommand\arraystretch{1.2}

% Table with easy to use footnotes
% \begin{threeparttable}
%    \begin{tabular} ... \end{tabular}
%    \begin{tablenotes}
%        \item \textit{Notes.}
%    \end{tablenotes}  
% \end{threeparttable}
\usepackage[flushleft]{threeparttable}
\setlength\labelsep{0pt}

% \toprule, \cmidrule, \bottomrule
\usepackage{booktabs}

% If tables are too narrow, fill columns using:
% \begin{tabularx}{\linewidth}{cols}
% col-types: X - center, L - left, R -right
% If you want relative scale for columns: 
% >{\hsize=.8\hsize}X/L/R
\usepackage{tabularx}
\newcolumntype{L}{>{\raggedright\arraybackslash}X}
\newcolumntype{R}{>{\raggedleft\arraybackslash}X}
\newcolumntype{C}{>{\centering\arraybackslash}X}

% Shorter multicolumn commands
\newcommand{\mcc}[1]{\multicolumn{1}{c@{}}{#1}}
\newcommand{\mcl}[1]{\multicolumn{1}{l@{}}{#1}}
\newcommand{\mcr}[1]{\multicolumn{1}{r@{}}{#1}}

% d column
\usepackage{dcolumn}
\newcolumntype{d}[1]{D..{#1}}

% Landscape table 
% \begin{landscape} \pagestyle{lscaped} table... \end{landscsape}
% \usepackage{pdflscape} - rotates page left-side up in pdf
% \usepackage{lscape} - does not rotate page, only figure/table

\usepackage{pdflscape}

% For landscape, fix page number location
\usepackage{fancyhdr}
\fancypagestyle{lscaped}{%
    \fancyhf{}
    \renewcommand{\headrulewidth}{0pt}
    \textnormal
    \fancyfoot{%
        \tikz[remember picture,overlay]
        \node[outer sep=2.5cm,above,rotate=90] at (current page.east) {\thepage};
    }
}
  

% ------------------------------------------------------------------------------

\addbibresource{references.bib}

\usepackage{adjustbox}
\usepackage{tabularx}
\usepackage{booktabs}
\usepackage{threeparttable}

% ------------------------------------------------------------------------------
\begin{document}

% ------------------------------------------------------------------------------
\maketitle
% ------------------------------------------------------------------------------

\begin{frame}{Spatial Spillovers}
    Effects of treatments at states, counties, etc. can spill over boundaries.
        
    \begin{itemize}
        \item e.g. large employer opening/closing in a county have positive or negative employment effects on nearby counties
    \end{itemize}
    
    Differences-in-differences that compare outcomes among treated units with ``control'' units are biased from the spillover of effects.
\end{frame}

\begin{frame}{Research Question}
    What are the sources of bias when researchers do not account for sptial spillover of treatment effects? 
        
    \begin{itemize}
        \item In what contexts are these biases particularly large?
        
        \item Do current simple adjustments work?
    \end{itemize}
\end{frame}

\begin{frame}{Preview of Results}
    There are two sources of bias:
        \begin{itemize}
            \item[1.] Treatment effects spillover onto control units
            \item[2.] Treatment effects spillover onto also treated units
        \end{itemize}

    Solutions:
        \begin{itemize}
            \item Fixing bias 1 is relatively easy and works better when you have relatively few treated units
            
            \item Bias 2 worsens as treatment locations bunch more (spatial correlation of treatment increases)
            
            \item However, spatial correlation of treatment decreases bias 1
        \end{itemize}
\end{frame}

% Just TE
\imageframe{../../figures/figure-map_te.png}

% TE + Spill Control
\imageframe{../../figures/figure-map_te_spill_control.png}

% TE + Spill Control + Spill Treat
\imageframe{../../figures/figure-map_te_spill_all.png}



% ------------------------------------------------------------------------------
\section{Theory}
% ------------------------------------------------------------------------------

\begin{frame}{Potential Outcomes Framework}
    For exposition, I will label units as counties. Assume all treatment occurs at the same time (2-periods or pre-post averages).\footnote{This form adapts the framework of \citet{Vazquez-Bare_2019} for the diff-in-diff setting.}
    
    \begin{itemize}
        \item Let $Y_{it}(D_i, h_i(\vec{D}))$ be the potential outcome of county $i \in \{ 1, \dots, N \}$ at time $t$ with treatment status $D_i \in \{0, 1\}$.
        
        \item $\vec{D} \in \{0,1\}^N$ is the vector of all units treatments.
        
        \item The function $h_i(\vec{D})$ maps the entire treatment vector into a scalar that completely determines the spillover. 
        
        \item In the world with no treatment, $\vec{0}$, we say $h_i(\vec{0}) = 0$.
    \end{itemize}
\end{frame}

\begin{frame}{Examples of $h_i(\vec{D})$}
    
    Examples of $h_i(\vec{D})$:
    
    \begin{itemize}
        \item \textbf{Treatment within $x$ miles:}
        
        $h_i(\vec{D})$ = 1 if there is a treated unit within $x$-miles of unit $i$ and = 0 otherwise
            
        \item \textbf{Average treatment among k-nearest neighbors:}
        
        $h_i(\vec{D}) = \sum_{j = 1}^k D_{n(i,j)} / k$ where $n(i,j)$ represents the j-th closest neighbor to unit $i$. 


    \end{itemize}
\end{frame}

\begin{frame}{Estimand of Interest}
    Estimand of Interest: \[ 
        \textcolor{green}{\tau_{\text{direct}}} \equiv \textcolor{green}{\mathbb{E}\left[ Y_{i,1}(1, 0) - Y_{i,1}(0, 0) \mid D_i = 1\right]}
    \]

    This is the direct effect in the absense of spillovers.
\end{frame}

\begin{frame}{Parallel Trends}
    I assume a modified version of the parallel trends assumption: 
    \[
        \mathbb{E}\left[ Y_{i,1}(0, 0) - Y_{i,0}(0, 0) \mid D_i = 1, \vec{D} = 0 \right] 
    \] \[
        \quad =
        \mathbb{E}\left[ Y_{i,1}(0, 0) - Y_{i,0}(0, 0) \mid D_i = 0, \vec{D} = 0 \right],
    \]

    \begin{itemize}
        \item In the complete absence of treatment (not just the absence of individual $i$'s treatment), the change in potential outcomes from period 0 to 1 would not depend on treatment status
    \end{itemize}
\end{frame}

\begin{frame}{What does Diff-in-Diff identify?}
    With the parallel trends assumption and random assignment of $D_i$, I decompose the diff-in-diff estimate as follows: 
        
    \begin{align*}
        &\mathbb{E}_i \left[ Y_{i1} - Y_{i0} \mid D_i = 1 \right] - \mathbb{E}_i \left[ Y_{i1} - Y_{i0} \mid D_i = 0 \right] \\
        &\quad = 
        \textcolor{green}{\mathbb{E}_i \left[ Y_{i1}(1, 0) - Y_{i1}(0, 0) \mid D_i = 1 \right]} \\
        &\quad\quad + 
        \textcolor{red}{\mathbb{E}_i \left[ Y_{i1}(1, h_i(\vec{D})) - Y_{i1}(1, 0) \mid D_i = 1 \right]} \\ 
        &\quad\quad - 
        \textcolor{purple}{\mathbb{E}_i \left[ Y_{i1}(0, h_i(\vec{D})) - Y_{i1}(0, 0) \mid D_i = 0 \right]} \\
        &\quad \equiv \textcolor{green}{\tau_{\text{direct}}} + \textcolor{red}{\tau_{\text{spillover, treated}}} - \textcolor{purple}{\tau_{\text{spillover, control}}}
    \end{align*}

\end{frame}

\begin{frame}{Signing the Bias}
    \[ 
        \textcolor{green}{\tau_{\text{direct}}} + \textcolor{red}{\tau_{\text{spillover, treated}}} - \textcolor{purple}{\tau_{\text{spillover, control}}}    
    \]

    \begin{table}
        \caption{Bias from Spillovers}
        \begin{tabular}{|l|cc|}
            \hline
            & + Spillover & - Spillover \\ \hline
            onto Treated & + Bias & - Bias \\
            onto Control & - Bias & + Bias \\
            \hline
        \end{tabular}
    \end{table}
\end{frame}

\begin{frame}{Example of Bias}
    \textbf{Treated Spillover:}
    
    $\mathbb{E}_i \left[ Y_{i1}(1, h_i(\vec{D})) - Y_{i1}(1, 0) \mid D_i = 1 \right] = -0.5$ if county $i$ is within 40 miles of a treated county and $0$ otherwise.
        
    \textbf{Control Spillover:}
    
    $\mathbb{E}_i \left[ Y_{i1}(0, h_i(\vec{D})) - Y_{i1}(0, 0) \mid D_i = 0 \right] = 1$ if county $i$ is within 40 miles of a treated county and $0$ otherwise.
        
    \begin{itemize}
        \item Then, the bias term is \begin{align*}
            -0.5 * \text{\% of treated within 40 miles of a treated county} \\
            -1 * \text{\% of control within 40 miles of a treated county}
        \end{align*}
    \end{itemize}
\end{frame}



% ------------------------------------------------------------------------------
\section{Simulations}
% ------------------------------------------------------------------------------

\begin{frame}{Monte Carlo Simulations}
    \begin{itemize}
        \item Simulated the following Data Generating Process \begin{align*}
            y_{it} &= -2 + \mu_t + \mu_i + \tau_{\text{direct}} D_{it} + \tau_{\text{spillover, control}} (1-D_{it}) \text{Near}_{it} \\ 
            &\quad + \tau_{\text{spillover, treated}} D_{it} \text{Near}_{it} + \varepsilon_{it}
        \end{align*}

        \item Observation $i$ is a U.S. county, year $t \in \{2000, \dots, 2019\}$, treatment $D_{it}$ turns on in 2010 and is assigned randomly or through a spatial process.
        
        \item $\text{Near}_{it}$ is an indicator for having your county centroid be within 40 miles of a treated unit. 
        
        \item $\tau_{\text{direct}} = 2$, $\tau_{\text{spillover, control}} = 1$ and $\tau_{\text{spillover, treat}} = -0.5$.
    \end{itemize}
\end{frame}

\begin{frame}{Source of Bias 1: Control Units}
    Simulation 1: 
    
    \begin{itemize}
        \item The first simulation changes the probability of treatment, $D_{it}$.
        
        \item Assume, for now, there is only spillover on the control.
        
        \item Estimate the following equation: \[ 
            y_{it} = \alpha + \mu_t + \mu_i + \tau D_{it} + \epsilon_{it}    
        \]
        
        \item The bias of our estimate is \textit{Bias} $= \hat{\tau} - 2$
    \end{itemize}


    
\end{frame}

\imageframe{../../figures/figure-bias_from_prob_treat_slides.png}


\begin{frame}{Solution: Removing ``contaminated'' controls}
    A common solution to the problem of spillover is to restimate without neighboring control units and see what happens to the bias.

    This simulation assumes the correct ``contaminated'' control units, i.e. only control units with $h_i(\vec{D}) \neq 0$, which is not possible for researchers.
\end{frame}

\imageframe{../../figures/figure-bias_fix_slides.png}


\begin{frame}{Source of Bias 2: Treated Units}
    Simulation 2:

    \begin{itemize}
        \item I now add in the bias from spillover of treated units on to other treated units.
        
        \item This source of bias only occurs when treated units are located next to eachother, so the magnitude of bias depends on spatial autocorrelation of $D_{it}$. 
        
        \item Using a method from geosciences, I generate correlated ``fields'' across the US. 
        
        \item Then, for the counties in the top 10\% of values (hot spots in the field), I increase the probability of treatment. 
        
        \item The unconditional probability remains fixed at 10\%.
    \end{itemize}
\end{frame}

\imageframe{../../figures/figure-krig_slides.png}

\imageframe{../../figures/figure-spcorr_high_map_te_spill_all.png}

\imageframe{../../figures/figure-spcorr_low_map_te_spill_all}

\imageframe{../../figures/figure-bias_from_spatial_autocorr_slides.png}

\imageframe{../../figures/figure-bias_fix_spatial_autocorr_slides.png}


\begin{frame}{Spillovers as estimand of interest}
    Until now, we assumed our estimand of interest is $\textcolor{green}{\tau_{\text{direct}}}$.

    However, the two other spillover effects are of interest as well:
    \begin{itemize}
        \item $\textcolor{purple}{\tau_{\text{spillover, control}}}$: Do the benefits of a treated county come at a cost to neighbor counties? 
        
        \item $\textcolor{red}{\tau_{\text{spillover, treated}}}$: Does the estimated effect change based on others treatment? (This is what you should consider if you are a policy maker)
    \end{itemize}

    To estimate the spillover effects, we have to parameterize $h_i(\vec{D})$ function and the potential outcomes function $Y_i(D_i, h_i(\vec{D}))$.
\end{frame}



% ------------------------------------------------------------------------------
\section{Application: Tennessee Valley Authority}
% ------------------------------------------------------------------------------

\begin{frame}{Tennessee Valley Authority}
    \citet{Kline_Moretti_2014} look at the long-run impacts of the Tennessee Valley Authority (TVA).

    \begin{itemize}
        \item The TVA was a large-scale federal investment started in 1934 that focused on improving manufacturing economy.
        
        \item The program focused on electrification through dams and transportation canals in order to improve the manufacturing economy
        
        \item Hundreds of dollars spent anually per person
    \end{itemize}
\end{frame}

\begin{frame}{Identification}
    \citet{Kline_Moretti_2014} run the county-level difference-in-differences  specification: 
    \begin{equation}\label{eq:tva_spillover}
        y_{c, 2000} - y_{c, 1940} = \alpha + \text{TVA}_c \tau + X_{c, 1940} \beta + (\varepsilon_{c, 2000} - \varepsilon_{c, 1940})
    \end{equation} 

    \begin{itemize}
        \item $y$ are outcomes for agricultural employment, manufacturing employment, and median family income.
        \item $\text{TVA}_c$ is the treatment variable
        \item $X_{c, 1940}$ allow for different long-term trends based on covariates in 1940. 
    \end{itemize}

    They trim the sample using a logit regression to predict treatment using $X_{c,1940}$ and keep control units in the top 75\% of predicted probability.
\end{frame}

\begin{frame}{Spillovers in the TVA Context}
    In our context, there is reason to believe spillovers can occur to nearby counties

    \begin{itemize}
        \item Agricultural employees might be drawn to hire wages for new manufacturing jobs in Tennessee Valley 
        
        \item Manufacturing jobs that would have been created in the control units in the absence of treatment might move to the Tennessee Valley
        
        \item Since manufacutring jobs pay higher than agricultural jobs, the above two spillover effects can cause changes in median family income
    \end{itemize}
\end{frame}

\begin{frame}{Specification including spillovers}
    \begin{equation}
        \Delta y_c = \alpha + \text{TVA}_i \tau + \sum_{d \in \text{Dist}} \text{Between}(d)\delta_d + X_{i, 1940} \beta + \Delta \varepsilon_c
    \end{equation} 

    \begin{itemize}
        \item $\text{Between}(d)$ is a set of indicators for being in the following distance bins (in miles) from the Tennessee Valley Authority: 
        \[ d \in \left\{ (0, 50], (50, 100], (100, 150] \right\} \]
    \end{itemize}
\end{frame}

\imageframe{../../figures/figure-tva-sample_slides.pdf}

\begin{frame}
\begin{table}[ht]
    \caption{Effects of Tennessee Valley Authority on Decadel Growth}
    \label{tab:tva}
    \renewcommand{\arraystretch}{1.2}

    \begin{adjustbox}{width = 1\textwidth, center}
        \begin{threeparttable}
            \begin{tabular}{@{} lc@{\extracolsep{20pt}}c@{\extracolsep{4pt}}ccc @{}}
                % Head
                \toprule

                & \multicolumn{1}{c}{\textbf{Diff-in-Diff}} & \multicolumn{4}{c}{\textbf{Diff-in-Diff with Spillovers}} \\ 
                \cmidrule{2-2} \cmidrule{3-6} 
                & & & TVA between & TVA between & TVA between \\ 
                \textit{Dependent Var.} & TVA & TVA & 0-50 mi. & 50-100 mi. & 100-150 mi. \\ 

                % Body
                \midrule
                
                
                 Agricultural employment     & $-0.0645^{***}$& $-0.0584^{***}$& $-0.0760^{***}$& $-0.0351^{***}$&  $-0.0179^{*}$ & $-0.0249^{***}$\\
                             &   $(0.0106)$   &   $(0.0099)$   &   $(0.0101)$   &   $(0.0134)$   &   $(0.0097)$   &   $(0.0090)$   \\
 Manufacturing employment    & $0.0625^{***}$ & $0.0588^{***}$ & $0.0518^{***}$ &    $-0.0057$   &    $-0.0099$   &    $-0.0220$   \\
                             &   $(0.0164)$   &   $(0.0157)$   &   $(0.0179)$   &   $(0.0210)$   &   $(0.0238)$   &   $(0.0139)$   \\
 Median family income        & $0.0242^{***}$ & $0.0214^{***}$ & $0.0217^{***}$ &    $0.0060$    &    $-0.0051$   &    $-0.0027$   \\
                             &   $(0.0066)$   &   $(0.0073)$   &   $(0.0083)$   &   $(0.0078)$   &   $(0.0062)$   &   $(0.0037)$   \\

                
                \bottomrule
            \end{tabular}
            
            % Notes 
            \begin{tablenotes}\footnotesize

                \item $^{*} p< 0.1$; $^{**} p < 0.05$; $^{***} p < 0.01$.
            \end{tablenotes}
        \end{threeparttable}
    \end{adjustbox}
\end{table}    

\end{frame}


























\end{document}
