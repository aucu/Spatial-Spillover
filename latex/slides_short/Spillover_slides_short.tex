\documentclass[aspectratio=169]{beamer}
% Metropolis Theme ------------------------------------------------------------------------------
\usetheme{metropolis} % Use metropolis theme


% Title ------------------------------------------------------------------------------
\title{Difference-in-Differences with Spatial Spillovers}
\date{\today}
\author{Kyle Butts}
% \institue{}

% Margins ----------------------------------------------------------------------

\usepackage[margin=1.25in]{geometry}

% AMS --------------------------------------------------------------------------

\usepackage{amsmath}
\usepackage{amsfonts}
\usepackage{amsthm}
\usepackage{graphicx}


% Line Spacing -----------------------------------------------------------------

\renewcommand{\baselinestretch}{1.5}


% Font -------------------------------------------------------------------------

\usepackage[T1]{fontenc}
\usepackage[default]{lato} % Lato as text font
% \usepackage[utopia, varg]{newtxmath}
% \renewcommand{\rmdefault}{futs} % Utopia as text font 

% Small adjustments to text kerning
\usepackage{microtype}

% Remove annoying over-full box warnings
\vfuzz2pt 
\hfuzz2pt


% Tikz support -----------------------------------------------------------------

\usepackage{tikz}


% Color Palette ----------------------------------------------------------------

\usepackage{xcolor}

% https://www.materialpalette.com/colors
\definecolor{dark-maroon}{HTML}{5D0F0D}
\definecolor{navyblue}{HTML}{0A3044}

% From Davidson Mackinnon
\definecolor{dm-blue}{HTML}{086fbd}
\definecolor{dm-red}{HTML}{ba3132}
\definecolor{dm-green}{HTML}{3f7e32}

% https://www.viget.com/articles/color-contrast/
\definecolor{purple}{HTML}{5601A4}
\definecolor{navy}{HTML}{0D3D56}
\definecolor{ruby}{HTML}{9a2515}
\definecolor{alice}{HTML}{107895}
\definecolor{daisy}{HTML}{EBC944}
\definecolor{coral}{HTML}{F26D21}
\definecolor{kelly}{HTML}{829356}
\definecolor{cranberry}{HTML}{E64173}
\definecolor{jet}{HTML}{131516}
\definecolor{asher}{HTML}{555F61}
\definecolor{slate}{HTML}{314F4F}


% Hyperlinks -------------------------------------------------------------------

\usepackage{hyperref}
\hypersetup{
    colorlinks= true,
    citecolor= dark-maroon,
    linkcolor= dark-maroon,
    filecolor= dark-maroon,      
    urlcolor= dark-maroon,
}


% Citations --------------------------------------------------------------------

% note, natbib provides better hyperlinking
\usepackage{natbib}
\bibliographystyle{econ-aea}


% Define Theorems --------------------------------------------------------------

% Put proper spacing after Theorem #. 
\newtheoremstyle{spacing}
{}%          Space above, empty = `usual value'
{}%          Space below
{}%  Body font
{}%          Indent amount (empty = no indent, \parindent = para indent)
{\bfseries\color{navyblue}}% Thm head font
{.}%         Punctuation after thm head
{2.5mm}%  Space after thm head: \newline = linebreak
{}%          Thm head spec

% note, theorem is the name that goes in \begin{} and Theorem is the name displayed as Theorem 1
\theoremstyle{spacing}
\newtheorem{theorem}{Theorem}
\newtheorem{proposition}{Proposition}
\newtheorem{assumption}{Assumption}
\newtheorem{example}{Example}


% Custom Math Definitions ------------------------------------------------------

\newcommand{\expec}[1]{\mathbb{E}\left[#1\right]}%
\newcommand{\condexpec}[2]{\mathbb{E}\left[#1 \ \vert \ #2\right]}%
\newcommand{\prob}[1]{\mathbb{P}\left[#1\right]}%
\newcommand{\var}[1]{\mathrm{Var}\left[#1\right]}%
\newcommand{\cov}[1]{\mathrm{Cov}\left[#1\right]}%
\newcommand{\one}{\mathbf{1}}


% Titlepage --------------------------------------------------------------------

% \maketitle
\usepackage{titling}
\usepackage{setspace}

% title
\pretitle{\begin{spacing}{1}\begin{flushleft}\huge}
\posttitle{\end{flushleft}\end{spacing}\vspace{-5mm}}
% author, note don't use \and 
\preauthor{\begin{flushleft}\LARGE}
\postauthor{\end{flushleft}\vspace{-7.5mm}}
% date
\predate{\begin{flushleft}\Large\color{asher}}
\postdate{\end{flushleft}\vspace{-5mm}}

% Abstract
\renewenvironment{abstract}
 {\noindent\rule{\linewidth}{.5pt}\noindent}
 {\noindent\rule{\linewidth}{.5pt}}

% alternative abstract
% \renewenvironment{abstract}
% {
%   \centerline {\large \bfseries \scshape \color{navyblue} Abstract}
%   \begin{quote}
% }
% {\end{quote}}


% Section and Subsection Styling -----------------------------------------------

\usepackage[explicit]{titlesec}

\titleformat{\section}
  {\Large \bf \color{navyblue}}
  {\thesection \,---}
  {0.25em}
  {#1}
  
\titleformat{\subsection}
  {\fontsize{11}{10}\it}
  {\thesubsection.}
  {1em}
  {#1}

% Don't number subsubsection
\setcounter{secnumdepth}{2}

% Footnote ---------------------------------------------------------------------

% Spacing between footnotes on same page
\addtolength{\footnotesep}{1mm}

% Space after footnote number
\let\oldfootnote\footnote
\renewcommand\footnote[1]{\oldfootnote{\ #1}}

% No footnote line
\renewcommand\footnoterule{}

% No supsercript in footer
\makeatletter
\renewcommand\@makefntext[1]{%
    \parindent 1em \noindent
    \hb@xt@1.8em{\hss\normalfont\@thefnmark.\hfill}#1
  }
\makeatother




% Enumerate/Itemize ------------------------------------------------------------

\usepackage{enumitem}
\setitemize{labelindent=0.5em,labelsep=0.25cm,leftmargin=*}
\setenumerate{labelindent=0.5em,labelsep=0.25cm,leftmargin=*}


% Table and Figure labelling ---------------------------------------------------

\usepackage{caption}

\DeclareCaptionLabelSeparator{threedash}{\,---\,}
\DeclareCaptionFont{navyblue}{\color{navyblue}}
\DeclareCaptionFont{jet}{\color{jet}}
\captionsetup[table]{format=plain, labelsep=threedash, font={navyblue, bf}}
\captionsetup[figure]{format=plain, labelsep=threedash, font={navyblue, bf}}

% Alternative: Left align captions
% \captionsetup[table]{labelfont=it, textfont={navyblue, bf}, labelsep=newline, justification=raggedright, singlelinecheck=off}
% \captionsetup[figure]{labelfont=it, textfont={navyblue, bf}, labelsep=newline, justification=raggedright, singlelinecheck=off}

% multifigure with \caption
% \begin{subfigure}\caption{} \end{subfigure}
\usepackage{subcaption}
\captionsetup[subfigure]{format=plain, font={jet, footnotesize, bf}}


% Tables -----------------------------------------------------------------------

% Fix \input with tables
% \input fails when \\ is at end of external .tex file

\makeatletter
\let\input\@@input
\makeatother

% Make tables/figures wider than \textwidth using:
% \begin{adjustbox}{width = 1.2\textwidth, center}
% \end{adjustbox}
\usepackage{adjustbox}

% Slighty more spacing between rows
\usepackage{array}
\renewcommand\arraystretch{1.2}

% Table with easy to use footnotes
% \begin{threeparttable}
%    \begin{tabular} ... \end{tabular}
%    \begin{tablenotes}
%        \item \textit{Notes.}
%    \end{tablenotes}  
% \end{threeparttable}
\usepackage[flushleft]{threeparttable}
\setlength\labelsep{0pt}

% \toprule, \cmidrule, \bottomrule
\usepackage{booktabs}

% If tables are too narrow, fill columns using:
% \begin{tabularx}{\linewidth}{cols}
% col-types: X - center, L - left, R -right
% If you want relative scale for columns: 
% >{\hsize=.8\hsize}X/L/R
\usepackage{tabularx}
\newcolumntype{L}{>{\raggedright\arraybackslash}X}
\newcolumntype{R}{>{\raggedleft\arraybackslash}X}
\newcolumntype{C}{>{\centering\arraybackslash}X}

% Shorter multicolumn commands
\newcommand{\mcc}[1]{\multicolumn{1}{c@{}}{#1}}
\newcommand{\mcl}[1]{\multicolumn{1}{l@{}}{#1}}
\newcommand{\mcr}[1]{\multicolumn{1}{r@{}}{#1}}

% d column
\usepackage{dcolumn}
\newcolumntype{d}[1]{D..{#1}}

% Landscape table 
% \begin{landscape} \pagestyle{lscaped} table... \end{landscsape}
% \usepackage{pdflscape} - rotates page left-side up in pdf
% \usepackage{lscape} - does not rotate page, only figure/table

\usepackage{pdflscape}

% For landscape, fix page number location
\usepackage{fancyhdr}
\fancypagestyle{lscaped}{%
    \fancyhf{}
    \renewcommand{\headrulewidth}{0pt}
    \textnormal
    \fancyfoot{%
        \tikz[remember picture,overlay]
        \node[outer sep=2.5cm,above,rotate=90] at (current page.east) {\thepage};
    }
}
  

% ------------------------------------------------------------------------------

\addbibresource{references.bib}

\usepackage{adjustbox}
\usepackage{tabularx}
\usepackage{booktabs}
\usepackage{threeparttable}
\usepackage{dcolumn} 

% Table Highlighting
\usepackage[beamer,customcolors]{hf-tikz}
\usetikzlibrary{calc}
\usetikzlibrary{fit,shapes.misc}

% To set the hypothesis highlighting boxes red.
\tikzset{hl/.style={
    set fill color=red!80!black!40,
    set border color=red!80!black,
  },
}
\newcommand\marktopleft[1]{%
    \tikz[overlay,remember picture] 
        \node (marker-#1-a) at (0,1.5ex) {};%
}
\newcommand\markbottomright[1]{%
    \tikz[overlay,remember picture] 
        \node (marker-#1-b) at (0,0) {};%
    \tikz[red, ultra thick, overlay, remember picture, inner sep=4pt]
        \node[draw, rectangle, fit=(marker-#1-a.center) (marker-#1-b.center)] {};%
}


% ------------------------------------------------------------------------------
\begin{document}

% ------------------------------------------------------------------------------
\maketitle
% ------------------------------------------------------------------------------

\begin{frame}{Spatial Spillovers}
    Researchers aim to estimate the \textbf{average treatment effect on the treated}: 
    \[
        \tau \equiv \mathbb{E} \left[ Y_{i1}(1) - Y_{i1}(0) \ \vert \ D_{i} = 1 \right]
    \]
    
    Estimation is complicated by \textbf{Spillover Effects}
    
    \vspace{5mm}
    \textbf{Spillover effects} are when effect of treatment extend over the treatment boundaries (states, counties, etc.). Example:
    
    \begin{itemize}
        \item A large employer opening/closing in a \textbf{treated} county have positive employment effects on \textit{nearby counties}
        
        \item Having nearby counties with factories raises wages and reduces effect of \textbf{treated} counties
    \end{itemize}
\end{frame}

\begin{frame}{Bias from Spatial Spillovers}
    
    \onslide<1->{
        The canonical difference-in-differences estimate is: 
        \only<1>{
            \[ 
                \hat{\tau} = \underbrace{\hat{\mathbb{E}} \left[ Y_{i1} - Y_{i0} \mid D_i = 1 \right]}_{\text{Counterfactual Trend} \ + \ \tau} - 
                \underbrace{\hat{\mathbb{E}} \left[ Y_{i1} - Y_{i0} \mid D_i = 0 \right]}_{\text{Counterfactual Trend}}
            \]
        }
        \only<2>{
            \[ 
                \hat{\tau} = \underbrace{\hat{\mathbb{E}} \left[ Y_{i1} - Y_{i0} \mid D_i = 1 \right]}_{\text{Counterfactual Trend} \ + \ \tau} - 
                \underbrace{\hat{\mathbb{E}} \left[ Y_{i1} - Y_{i0} \mid D_i = 0 \right]}_{\substack{\text{Counterfactual Trend} \\[2mm] \ + \ \text{\color{purple} Spillover on Control}}}
            \]
        }
        \only<3>{
            \[ 
                \hat{\tau} = \underbrace{\hat{\mathbb{E}} \left[ Y_{i1} - Y_{i0} \mid D_i = 1 \right]}_{\substack{\text{Counterfactual Trend} \ + \ \tau \\[2mm] \ + \ \text{\color{red} Spillover on Treated}}} - 
                \underbrace{\hat{\mathbb{E}} \left[ Y_{i1} - Y_{i0} \mid D_i = 0 \right]}_{\substack{\text{Counterfactual Trend} \\[2mm] \ + \ \text{\color{purple} Spillover on Control}}}
            \]
        }
    } 

    Two problems in presence of spillover effects:
    
    
    \begin{itemize}
        \onslide<2->{
            \item {\bf \color{purple} Spillover onto Control Units:} 
            
            Nearby ``control'' units fail to estimate counterfactual trends because they are affected by treatment
        }
        
        \onslide<3->{
            \vspace{2.5mm}
            \item {\bf \color{red} Spillover onto other Treated Units:} 
            
            Treated units are also affected by nearby units and therefore combines ``direct'' effects with spillover effects
        }
    \end{itemize}

\end{frame}


\begin{frame}{Econometric Contribution}

    I formalize spillovers into a potential outcomes framework:
 
    \begin{citecolor}
        [\citet{Clarke_2017}, \citet{Berg_Streitz_2019}, and \citet{Verbitsky-Savitz_Raudenbush_2012}]
    \end{citecolor}

    \begin{itemize}
        \vspace{2.5mm}
        \item I decompose the difference-in-differences estimator into three parts: Direct Effect of Treatment, Spillover onto Treated Units, Spillover onto Control Units
        
        \vspace{2.5mm}
        \item Show that an indicator for being close to treated units remove \textit{all bias} so long as the indicator contains all units affected by spillovers
        
        \vspace{2.5mm}
        \item `Rings' are able to estimate spillover effects non-parametrically while still removing all bias
    \end{itemize} 

\end{frame}


% Grey out overlays
\setbeamercovered{transparent}



% ------------------------------------------------------------------------------
\section{Theory}
% ------------------------------------------------------------------------------

\begin{frame}{Potential Outcomes Framework}
    
    $Y_{it}(D_i, \textcolor{blue}{h(\vec{D}, i)})$ is the potential outcome of county $i \in \{ 1, \dots, N \}$ at time $t$ with treatment status $D_i \in \{0, 1\}$.

    The function $\textcolor{blue}{h(\vec{D}, i)}$ maps the entire treatment vector into an `exposure mapping'
\end{frame}

\begin{frame}{Examples of $h_i(\vec{D})$}
    \textbf{Treatment within $x$ miles:}
        
    $\textcolor{blue}{h(\vec{D}, i)} = max_j \ 1(d(i, j) \leq x)$ where $d(i,j)$ is the distance between counties $i$ and $j$. 

    \begin{itemize}
        \item e.g. library access where $x$ is the maximum distance people will travel
        
        \item Spillovers are non-additive
    \end{itemize}
\end{frame}

\imageframe{../../figures/figure-spill_within_large.png}

\begin{frame}{Examples of $h_i(\vec{D})$}

    \textbf{Treatment within $x$ miles:}
        
    $\textcolor{blue}{h(\vec{D}, i)} = max_j \ 1(d(i, j) \leq x)$ where $d(i,j)$ is the distance between counties $i$ and $j$. 

    \begin{itemize}
        \item e.g. library access where $x$ is the maximum distance people will travel
        
        \item Spillovers are non-additive
    \end{itemize}
    
    \vspace{2.5mm}
    \textbf{Number of Treated within $x$ miles:}
    
    $\textcolor{blue}{h(\vec{D}, i)} = \sum_{j = 1}^k 1(d(i, j) \leq x)$. 

    \begin{itemize}
        \item e.g. large factories opening
        
        \item Agglomeration economies suggest spillovers are additive
    \end{itemize}


\end{frame}


\imageframe{../../figures/figure-spill_within_large_additive.png}


\begin{frame}{What does Diff-in-Diff identify?}
    With the parallel trends assumption and a random sample, I decompose the difference-in-differences estimate as follows: 
        
    \begin{align*}
        \mathbb{E}\left[\hat{\tau}\right] &= \underbrace{\mathbb{E} \left[ Y_{i1} - Y_{i0} \mid D_i = 1 \right] - \mathbb{E} \left[ Y_{i1} - Y_{i0} \mid D_i = 0 \right]}_{\text{Difference-in-Differences}} \\[3mm]
        \pause&= 
        \textcolor{green}{\mathbb{E} \left[ Y_{i1}(1, 0) - Y_{i1}(0, 0) \mid D_i = 1 \right]} \\
        &\quad + \quad 
        \textcolor{red}{\mathbb{E} \left[ Y_{i1}(1, h_i(\vec{D})) - Y_{i1}(1, 0) \mid D_i = 1 \right]} \\ 
        &\quad - \quad  
        \textcolor{purple}{\mathbb{E} \left[ Y_{i1}(0, h_i(\vec{D})) - Y_{i1}(0, 0) \mid D_i = 0 \right]} \\[3mm]
        &= \textcolor{green}{\tau_{\text{direct}}} + \textcolor{red}{\tau_{\text{spillover, treated}}} - \textcolor{purple}{\tau_{\text{spillover, control}}}
    \end{align*}

\end{frame}


% ------------------------------------------------------------------------------
\section{Estimation with Spillovers}
% ------------------------------------------------------------------------------

\begin{frame}{Spillovers as estimand of interest}
    Until now, we assumed our estimand of interest is $\textcolor{green}{\tau_{\text{direct}}}$.
    
    However, the two other spillover effects are of interest as well:
    \begin{itemize}
        \item $\textcolor{purple}{\tau_{\text{spillover, control}}}$: Do the benefits of a treated county come at a cost to neighbor counties? 
        
        \item $\textcolor{red}{\tau_{\text{spillover, treated}}}$: Does the estimated effect change based on others treatment? (This is what you should consider if you are a policy maker)
    \end{itemize}
    
    To estimate the spillover effects, we have to parameterize $h(\vec{D}, i)$ function and the potential outcomes function $Y_i(D_i, h(\vec{D}, i))$.
\end{frame}

\begin{frame}{Robustness to Misspecification}
    I find that an indicator for being Within $x$ miles from treated area interacted with treatment status will remove \textbf{all bias} so long as the indicator contains all the affected units.

    \begin{itemize}
        \item Indicator will estimate the average spillover effect on treated and control units and remove these from estimate, $\hat{\tau}$
    \end{itemize}
\end{frame}

\begin{frame}{Estimation of Spillover Effects}
    In a lot of settings, estimating the spillover effects are also an estimand of interest.

    A set of concentring `rings' around treatment perform best for estimating spillover effects
\end{frame}

\imageframe{../../figures/figure-spill_ring.png}

\begin{frame}{Benefits of Rings}

    \begin{itemize}
        \item Still remove all bias from the treatment effect estimate
        
        \item Can trace out how spillovers spread over distance
        
        \item If spillovers are additive in the number of nearby treated units, then an additive version of rings should be used (but this loses the bias-removal property)
    \end{itemize} 
    

\end{frame}


\section{Application in Urban Economics}


\begin{frame}{Urban Economics}

    I apply framework to place-based analyses in Urban Economics

    \begin{itemize}
        \item Revisit \begin{citecolor}\citet{Kline_Moretti_2014a}\end{citecolor} analysis of the Tennessee Valley Authority
        
        \begin{itemize}
            \vspace{2.5mm}
            \item The local effect estimate is contaminated by spillover effects to neighboring counties \begin{citecolor}\citep{Kline_Moretti_2014b}\end{citecolor}
            
            \vspace{2.5mm}
            \item Large scale manufacturing investment creates an `urban shadow' \begin{citecolor}\citep{Cuberes_Desmet_Rappaport_2021,Fujita_Krugman_Venables_2001}\end{citecolor}
        \end{itemize}
        
        \vspace{2.5mm}
        \item Discuss how framework can reconcile conflicting findings on effect of federal Empowerment Zones
        
        \begin{itemize}
            \vspace{2.5mm}
            \item Identification Strategy of using far-away rejected applicants \begin{citecolor}\citep{Busso_Gregory_Kline_2013} \end{citecolor} vs. census tracts within 1000 feet of empowerment zone \begin{citecolor}\citep{Neumark_Kolko_2010}\end{citecolor}
        \end{itemize}
    \end{itemize}

\end{frame}






\section{Conclusion}

\begin{frame}{Conclusion}
    \begin{itemize}
        \item I decomposed the TWFE estimate into the direct effect and two spillover terms
        
        \item I showed that a set of concentric rings removes two spillover terms from treatment effect estimate and models spillovers well
        
        \item For place-based policies, I show the importance of considering spatial spillovers when estimating treatment effects
    \end{itemize}
\end{frame}



\end{document}