\documentclass[aspectratio=169]{beamer}
% Metropolis Theme ------------------------------------------------------------------------------
\usetheme{metropolis} % Use metropolis theme


% Title ------------------------------------------------------------------------------
\title{Difference-in-Differences with Spatial Spillovers}
\date{\today}
\author{Kyle Butts}
% \institue{}

% Margins ----------------------------------------------------------------------

\usepackage[margin=1.25in]{geometry}

% AMS --------------------------------------------------------------------------

\usepackage{amsmath}
\usepackage{amsfonts}
\usepackage{amsthm}
\usepackage{graphicx}


% Line Spacing -----------------------------------------------------------------

\renewcommand{\baselinestretch}{1.5}


% Font -------------------------------------------------------------------------

\usepackage[T1]{fontenc}
\usepackage[default]{lato} % Lato as text font
% \usepackage[utopia, varg]{newtxmath}
% \renewcommand{\rmdefault}{futs} % Utopia as text font 

% Small adjustments to text kerning
\usepackage{microtype}

% Remove annoying over-full box warnings
\vfuzz2pt 
\hfuzz2pt


% Tikz support -----------------------------------------------------------------

\usepackage{tikz}


% Color Palette ----------------------------------------------------------------

\usepackage{xcolor}

% https://www.materialpalette.com/colors
\definecolor{dark-maroon}{HTML}{5D0F0D}
\definecolor{navyblue}{HTML}{0A3044}

% From Davidson Mackinnon
\definecolor{dm-blue}{HTML}{086fbd}
\definecolor{dm-red}{HTML}{ba3132}
\definecolor{dm-green}{HTML}{3f7e32}

% https://www.viget.com/articles/color-contrast/
\definecolor{purple}{HTML}{5601A4}
\definecolor{navy}{HTML}{0D3D56}
\definecolor{ruby}{HTML}{9a2515}
\definecolor{alice}{HTML}{107895}
\definecolor{daisy}{HTML}{EBC944}
\definecolor{coral}{HTML}{F26D21}
\definecolor{kelly}{HTML}{829356}
\definecolor{cranberry}{HTML}{E64173}
\definecolor{jet}{HTML}{131516}
\definecolor{asher}{HTML}{555F61}
\definecolor{slate}{HTML}{314F4F}


% Hyperlinks -------------------------------------------------------------------

\usepackage{hyperref}
\hypersetup{
    colorlinks= true,
    citecolor= dark-maroon,
    linkcolor= dark-maroon,
    filecolor= dark-maroon,      
    urlcolor= dark-maroon,
}


% Citations --------------------------------------------------------------------

% note, natbib provides better hyperlinking
\usepackage{natbib}
\bibliographystyle{econ-aea}


% Define Theorems --------------------------------------------------------------

% Put proper spacing after Theorem #. 
\newtheoremstyle{spacing}
{}%          Space above, empty = `usual value'
{}%          Space below
{}%  Body font
{}%          Indent amount (empty = no indent, \parindent = para indent)
{\bfseries\color{navyblue}}% Thm head font
{.}%         Punctuation after thm head
{2.5mm}%  Space after thm head: \newline = linebreak
{}%          Thm head spec

% note, theorem is the name that goes in \begin{} and Theorem is the name displayed as Theorem 1
\theoremstyle{spacing}
\newtheorem{theorem}{Theorem}
\newtheorem{proposition}{Proposition}
\newtheorem{assumption}{Assumption}
\newtheorem{example}{Example}


% Custom Math Definitions ------------------------------------------------------

\newcommand{\expec}[1]{\mathbb{E}\left[#1\right]}%
\newcommand{\condexpec}[2]{\mathbb{E}\left[#1 \ \vert \ #2\right]}%
\newcommand{\prob}[1]{\mathbb{P}\left[#1\right]}%
\newcommand{\var}[1]{\mathrm{Var}\left[#1\right]}%
\newcommand{\cov}[1]{\mathrm{Cov}\left[#1\right]}%
\newcommand{\one}{\mathbf{1}}


% Titlepage --------------------------------------------------------------------

% \maketitle
\usepackage{titling}
\usepackage{setspace}

% title
\pretitle{\begin{spacing}{1}\begin{flushleft}\huge}
\posttitle{\end{flushleft}\end{spacing}\vspace{-5mm}}
% author, note don't use \and 
\preauthor{\begin{flushleft}\LARGE}
\postauthor{\end{flushleft}\vspace{-7.5mm}}
% date
\predate{\begin{flushleft}\Large\color{asher}}
\postdate{\end{flushleft}\vspace{-5mm}}

% Abstract
\renewenvironment{abstract}
 {\noindent\rule{\linewidth}{.5pt}\noindent}
 {\noindent\rule{\linewidth}{.5pt}}

% alternative abstract
% \renewenvironment{abstract}
% {
%   \centerline {\large \bfseries \scshape \color{navyblue} Abstract}
%   \begin{quote}
% }
% {\end{quote}}


% Section and Subsection Styling -----------------------------------------------

\usepackage[explicit]{titlesec}

\titleformat{\section}
  {\Large \bf \color{navyblue}}
  {\thesection \,---}
  {0.25em}
  {#1}
  
\titleformat{\subsection}
  {\fontsize{11}{10}\it}
  {\thesubsection.}
  {1em}
  {#1}

% Don't number subsubsection
\setcounter{secnumdepth}{2}

% Footnote ---------------------------------------------------------------------

% Spacing between footnotes on same page
\addtolength{\footnotesep}{1mm}

% Space after footnote number
\let\oldfootnote\footnote
\renewcommand\footnote[1]{\oldfootnote{\ #1}}

% No footnote line
\renewcommand\footnoterule{}

% No supsercript in footer
\makeatletter
\renewcommand\@makefntext[1]{%
    \parindent 1em \noindent
    \hb@xt@1.8em{\hss\normalfont\@thefnmark.\hfill}#1
  }
\makeatother




% Enumerate/Itemize ------------------------------------------------------------

\usepackage{enumitem}
\setitemize{labelindent=0.5em,labelsep=0.25cm,leftmargin=*}
\setenumerate{labelindent=0.5em,labelsep=0.25cm,leftmargin=*}


% Table and Figure labelling ---------------------------------------------------

\usepackage{caption}

\DeclareCaptionLabelSeparator{threedash}{\,---\,}
\DeclareCaptionFont{navyblue}{\color{navyblue}}
\DeclareCaptionFont{jet}{\color{jet}}
\captionsetup[table]{format=plain, labelsep=threedash, font={navyblue, bf}}
\captionsetup[figure]{format=plain, labelsep=threedash, font={navyblue, bf}}

% Alternative: Left align captions
% \captionsetup[table]{labelfont=it, textfont={navyblue, bf}, labelsep=newline, justification=raggedright, singlelinecheck=off}
% \captionsetup[figure]{labelfont=it, textfont={navyblue, bf}, labelsep=newline, justification=raggedright, singlelinecheck=off}

% multifigure with \caption
% \begin{subfigure}\caption{} \end{subfigure}
\usepackage{subcaption}
\captionsetup[subfigure]{format=plain, font={jet, footnotesize, bf}}


% Tables -----------------------------------------------------------------------

% Fix \input with tables
% \input fails when \\ is at end of external .tex file

\makeatletter
\let\input\@@input
\makeatother

% Make tables/figures wider than \textwidth using:
% \begin{adjustbox}{width = 1.2\textwidth, center}
% \end{adjustbox}
\usepackage{adjustbox}

% Slighty more spacing between rows
\usepackage{array}
\renewcommand\arraystretch{1.2}

% Table with easy to use footnotes
% \begin{threeparttable}
%    \begin{tabular} ... \end{tabular}
%    \begin{tablenotes}
%        \item \textit{Notes.}
%    \end{tablenotes}  
% \end{threeparttable}
\usepackage[flushleft]{threeparttable}
\setlength\labelsep{0pt}

% \toprule, \cmidrule, \bottomrule
\usepackage{booktabs}

% If tables are too narrow, fill columns using:
% \begin{tabularx}{\linewidth}{cols}
% col-types: X - center, L - left, R -right
% If you want relative scale for columns: 
% >{\hsize=.8\hsize}X/L/R
\usepackage{tabularx}
\newcolumntype{L}{>{\raggedright\arraybackslash}X}
\newcolumntype{R}{>{\raggedleft\arraybackslash}X}
\newcolumntype{C}{>{\centering\arraybackslash}X}

% Shorter multicolumn commands
\newcommand{\mcc}[1]{\multicolumn{1}{c@{}}{#1}}
\newcommand{\mcl}[1]{\multicolumn{1}{l@{}}{#1}}
\newcommand{\mcr}[1]{\multicolumn{1}{r@{}}{#1}}

% d column
\usepackage{dcolumn}
\newcolumntype{d}[1]{D..{#1}}

% Landscape table 
% \begin{landscape} \pagestyle{lscaped} table... \end{landscsape}
% \usepackage{pdflscape} - rotates page left-side up in pdf
% \usepackage{lscape} - does not rotate page, only figure/table

\usepackage{pdflscape}

% For landscape, fix page number location
\usepackage{fancyhdr}
\fancypagestyle{lscaped}{%
    \fancyhf{}
    \renewcommand{\headrulewidth}{0pt}
    \textnormal
    \fancyfoot{%
        \tikz[remember picture,overlay]
        \node[outer sep=2.5cm,above,rotate=90] at (current page.east) {\thepage};
    }
}
  

% ------------------------------------------------------------------------------

\addbibresource{references.bib}

\usepackage{adjustbox}
\usepackage{tabularx}
\usepackage{booktabs}
\usepackage{threeparttable}
\usepackage{dcolumn} 

% Table Highlighting
\usepackage[beamer,customcolors]{hf-tikz}
\usetikzlibrary{calc}
\usetikzlibrary{fit,shapes.misc}

% To set the hypothesis highlighting boxes red.
\tikzset{hl/.style={
    set fill color=red!80!black!40,
    set border color=red!80!black,
  },
}
\newcommand\marktopleft[1]{%
    \tikz[overlay,remember picture] 
        \node (marker-#1-a) at (0,1.5ex) {};%
}
\newcommand\markbottomright[1]{%
    \tikz[overlay,remember picture] 
        \node (marker-#1-b) at (0,0) {};%
    \tikz[red, ultra thick, overlay, remember picture, inner sep=4pt]
        \node[draw, rectangle, fit=(marker-#1-a.center) (marker-#1-b.center)] {};%
}


% ------------------------------------------------------------------------------
\begin{document}

% ------------------------------------------------------------------------------
\maketitle
% ------------------------------------------------------------------------------

\begin{frame}{Spatial Spillovers}
    Researchers aim to estimate the average treatment effect on the treated: 
    \[
        \tau \equiv \mathbb{E} \left[ Y_{i1}(1) - Y_{i1}(0) \ \vert \ D_{i} = 1 \right]
    \]
    
    \vspace{5mm}
    Spillover effects are when effect of treatment extend over boundaries such as states, counties, etc.
    
    \begin{itemize}
        \item e.g. large employer opening/closing in a county have positive employment effects on nearby counties
    \end{itemize}
\end{frame}


\begin{frame}{Bias from Spatial Spillovers}
    
    \onslide<1->{
        The canonical difference-in-differences estimate is: 
        \only<1>{
            \[ 
                \hat{\tau} = \underbrace{\hat{\mathbb{E}} \left[ Y_{i1} - Y_{i0} \mid D_i = 1 \right]}_{\text{Counterfactual Trend} \ + \ \tau} - 
                \underbrace{\hat{\mathbb{E}} \left[ Y_{i1} - Y_{i0} \mid D_i = 0 \right]}_{\text{Counterfactual Trend}}
            \]
        }
        \only<2>{
            \[ 
                \hat{\tau} = \underbrace{\hat{\mathbb{E}} \left[ Y_{i1} - Y_{i0} \mid D_i = 1 \right]}_{\text{Counterfactual Trend} \ + \ \tau} - 
                \underbrace{\hat{\mathbb{E}} \left[ Y_{i1} - Y_{i0} \mid D_i = 0 \right]}_{\substack{\text{Counterfactual Trend} \\[2mm] \ + \ \text{\color{purple} Spillover on Control}}}
            \]
        }
        \only<3>{
            \[ 
                \hat{\tau} = \underbrace{\hat{\mathbb{E}} \left[ Y_{i1} - Y_{i0} \mid D_i = 1 \right]}_{\substack{\text{Counterfactual Trend} \ + \ \tau \\[2mm] \ + \ \text{\color{red} Spillover on Treated}}} - 
                \underbrace{\hat{\mathbb{E}} \left[ Y_{i1} - Y_{i0} \mid D_i = 0 \right]}_{\substack{\text{Counterfactual Trend} \\[2mm] \ + \ \text{\color{purple} Spillover on Control}}}
            \]
        }
    } 

    Two problems in presence of spillover effects:
    
    
    \begin{itemize}
        \onslide<2->{
            \item {\bf \color{purple} Spillover onto Control Units:} 
            
            Nearby ``control'' units fail to estimate counterfactual trends because they are affected by treatment
        }
        
        \onslide<3->{
            \vspace{2.5mm}
            \item {\bf \color{red} Spillover onto other Treated Units:} 
            
            Treated units are also affected by nearby units and therefore combines ``direct'' effects with spillover effects
        }
    \end{itemize}

\end{frame}



% Grey out overlays
\setbeamercovered{transparent}


\begin{frame}{Contribution}

    {\footnotesize
    
    \underline{Difference-in-Differences Estimation with Spillovers} 
    
    \begin{citecolor}
        [\citet{Clarke_2017}, \citet{Berg_Streitz_2019}, and \citet{Verbitsky-Savitz_Raudenbush_2012}]
    \end{citecolor}
    
    \begin{itemize}
        \item I generalize the work by deriving the bias in terms of potential outcomes and show how to estimate treatment effects by removing spillover effects
    \end{itemize}


    \pause
    \underline{Potential Outcome Framework for Spillovers}
    
    \begin{citecolor}
        [\citet{Miguel_Kremer_2004}, \citet{Vazquez-Bare_2019}, \citet{Angelucci_DiMaro_2016}, \citet{Angrist_2014}]        
    \end{citecolor}

    \begin{itemize}
        \item I focus on a spatial setting whereas these papers consider spillovers in network clusters
    \end{itemize}


    \pause
    \underline{Place-based Policies}

    \begin{citecolor}
        [\citet{Kline_Moretti_2014b}, \citet{Kline_Moretti_2014a}, \citet{Busso_Gregory_Kline_2013}, and \citet{Neumark_Kolko_2010}]
    \end{citecolor}

    \begin{itemize}
        \item I highlight the need to control for general equilibrium effects to properly estimate local effects of policies
    \end{itemize}
    }
\end{frame}


% ------------------------------------------------------------------------------
\section{Theory}
% ------------------------------------------------------------------------------


\begin{frame}{Potential Outcomes Framework}
    For exposition, I will label units as counties. Assume all treatment occurs at the same time (2-periods or pre-post averages).\footnote{I extend this into an event study framework in the paper, but the intuition is the same as in the $2 \times 2$ setting.}
    
    \begin{itemize}
        \item $Y_{it}(D_i, \textcolor{blue}{h(\vec{D}, i)})$ is the potential outcome of county $i \in \{ 1, \dots, N \}$ at time $t$ with treatment status $D_i \in \{0, 1\}$.
        
        \pause
        \item $\vec{D} \in \{0,1\}^N$ is the vector of all units treatments.
        
        \pause
        \item The function $\textcolor{blue}{h(\vec{D}, i)}$ maps the entire treatment vector into an `exposure mapping' which can be a scalar or a vector.
        
        \pause
        \item No exposure is when $\textcolor{blue}{h(\vec{D}, i)} = \vec{0}$.
    \end{itemize}
\end{frame}

% Grey out
% \setbeamercovered{transparent}

\begin{frame}{Examples of $h_i(\vec{D})$}
    
    Examples of $h_i(\vec{D})$:
    
    \begin{itemize}
        \item \textbf{Treatment within $x$ miles:}
        
        $\textcolor{blue}{h(\vec{D}, i)} = max_j \ 1(d(i, j) \leq x)$ where $d(i,j)$ is the distance between counties $i$ and $j$. 

        \begin{itemize}
            \item e.g. library access where $x$ is the maximum distance people will travel
            
            \item Spillovers are non-additive
        \end{itemize}

    \end{itemize}
\end{frame}

\imageframe{../../figures/figure-spill_within_large.png}

\begin{frame}{Examples of $h_i(\vec{D})$}
    
    Examples of $h_i(\vec{D})$:
    
    \begin{itemize}
        \item \textbf{Treatment within $x$ miles:}
        
        $\textcolor{blue}{h(\vec{D}, i)} = max_j \ 1(d(i, j) \leq x)$ where $d(i,j)$ is the distance between counties $i$ and $j$. 

        \begin{itemize}
            \item e.g. library access where $x$ is the maximum distance people will travel
            
            \item Spillovers are non-additive
        \end{itemize}
        
        \vspace{2.5mm}
        \item \textbf{Number of Treated within $x$ miles:}
        
        $\textcolor{blue}{h(\vec{D}, i)} = \sum_{j = 1}^k 1(d(i, j) \leq x)$. 

        \begin{itemize}
            \item e.g. large factories opening
            
            \item Agglomeration economies suggest spillovers are additive
        \end{itemize}

    \end{itemize}
\end{frame}


\imageframe{../../figures/figure-spill_within_large_additive.png}


\begin{frame}{Estimand of Interest}
    Estimand of Interest: \[ 
        \textcolor{green}{\tau_{\text{direct}}} \equiv \textcolor{green}{\mathbb{E}\left[ Y_{i,1}(1, 0) - Y_{i,1}(0, 0) \mid D_i = 1\right]}
    \]

    \vspace{10mm}
    This is the direct effect in the absense of exposure to spillovers.
\end{frame}


\begin{frame}{Parallel Trends}
    I assume a modified version of the parallel counterfactual trends assumption: 
    \begin{align*}
        &\mathbb{E}\big[ \underbrace{Y_{i,1}(0, \textcolor{blue}{\vec{0}}) - Y_{i,0}(0, \textcolor{blue}{\vec{0}})}_{\text{Counterfactual Trend}} \mid D_i = 1 \big] \\
        = \ &\mathbb{E}\big[ \underbrace{Y_{i,1}(0, \textcolor{blue}{\vec{0}}) - Y_{i,0}(0, \textcolor{blue}{\vec{0}})}_{\text{Counterfactual Trend}} \mid D_i = 0 \big],
    \end{align*}

    \vspace{5mm}
    In the complete absence of treatment (not just the absence of individual $i$'s treatment), the change in potential outcomes from period 0 to 1 would not depend on treatment status
    
\end{frame}

\begin{frame}{What does Diff-in-Diff identify?}
    With the parallel trends assumption and random assignment of $D_i$, I decompose the diff-in-diff estimate as follows: 
        
    \begin{align*}
        \mathbb{E}\left[\hat{\tau}\right] &= \underbrace{\mathbb{E} \left[ Y_{i1} - Y_{i0} \mid D_i = 1 \right] - \mathbb{E} \left[ Y_{i1} - Y_{i0} \mid D_i = 0 \right]}_{\text{Difference-in-Differences}} \\[3mm]
        \pause&= 
        \textcolor{green}{\mathbb{E} \left[ Y_{i1}(1, 0) - Y_{i1}(0, 0) \mid D_i = 1 \right]} \\
        &\quad + \quad 
        \textcolor{red}{\mathbb{E} \left[ Y_{i1}(1, h_i(\vec{D})) - Y_{i1}(1, 0) \mid D_i = 1 \right]} \\ 
        &\quad - \quad  
        \textcolor{purple}{\mathbb{E} \left[ Y_{i1}(0, h_i(\vec{D})) - Y_{i1}(0, 0) \mid D_i = 0 \right]} \\[3mm]
        &= \textcolor{green}{\tau_{\text{direct}}} + \textcolor{red}{\tau_{\text{spillover, treated}}} - \textcolor{purple}{\tau_{\text{spillover, control}}}
    \end{align*}

\end{frame}

\begin{frame}{Biased Estimate for $\tau_{\text{direct}}$}
    \[ 
        \mathbb{E}[\hat{\tau}_{\text{diff-in-diff}}] - \textcolor{green}{\tau_{\text{direct}}} = \textcolor{red}{\tau_{\text{spillover, treated}}} - \textcolor{purple}{\tau_{\text{spillover, control}}}    
    \]
\end{frame}




% ------------------------------------------------------------------------------
\section{Estimation}
% ------------------------------------------------------------------------------

\begin{frame}{Problems with "Dropping Bad Controls"}
    It is common in empirical applications to drop control units near the treated units when estimating the direct effect of treatment.

    This is not recommended for two reasons:

    \begin{itemize}
        \item[1.] If not all control units that experience spillover effects are removed, then bias remains
        
        \item[2.] There is a second source of bias, the spillover effects on treated units, that still remains
    \end{itemize}

    

\end{frame}


% ------------------------------------------------------------------------------
\section{Estimation with Spillovers}
% ------------------------------------------------------------------------------

\begin{frame}{Spillovers as estimand of interest}
    Until now, we assumed our estimand of interest is $\textcolor{green}{\tau_{\text{direct}}}$.
    
    However, the two other spillover effects are of interest as well:
    \begin{itemize}
        \item $\textcolor{purple}{\tau_{\text{spillover, control}}}$: Do the benefits of a treated county come at a cost to neighbor counties? 
        
        \item $\textcolor{red}{\tau_{\text{spillover, treated}}}$: Does the estimated effect change based on others treatment? (This is what you should consider if you are a policy maker)
    \end{itemize}
    
    To estimate the spillover effects, we have to parameterize $h(\vec{D}, i)$ function and the potential outcomes function $Y_i(D_i, h(\vec{D}, i))$.
\end{frame}

\begin{frame}{Robustness to Misspecification}
    Generate data using the same data-generating process as before but with different spillover functions:

    \[ 
        y_{\it} = \mu_t + \mu_i + 2 D_{it} + \beta_{\text{spill,control}} * (1-D_{it}) h(\vec{D}, i) + \varepsilon_{it}
    \]


    Then, I estimate each data-generating process using (potentially) misspecified $\tilde{h}(\vec{D}, i)$ and report the average estimate bias.

\end{frame}

\begin{frame}{Results}
    \textbf{I find that an indicator for being Within $x$ miles from treated area will remove all bias so long as the indicator contains all the affected units.}

\end{frame}

\begin{frame}{Estimation of Spillover Effects}
    In a lot of settings, estimating the spillover effects are also an estimand of interest.

    I repeat the same exercise and estimate the spillover effects for each control unit, $\hat{\beta}_{\text{spill, control}} * \tilde{h}(\vec{D}, i)$.

    Then calculate \[ 
        1 - \frac{
                \overbrace{\sum_{i: D_i = 0} (\beta_{\text{spill, control}} h(\vec{D}, i) - \hat{\beta}_{\text{spill, control}} \tilde{h}(\vec{D}, i))^2}^{\text{Mean Square Prediction Error}}
            }{
                \underbrace{\sum_{i: D_i = 0} (\beta_{\text{spill, control}} h(\vec{D}, i))^2}_{\text{Normalization}}
            }    
    \]

    This gives the proportion of spillovers explained by $\tilde{h}(\vec{D}, i)$

\end{frame}

\imageframe{../../figures/figure-spill_ring.png}

\begin{frame}{Results}
    \textbf{Rings perform best at estimating spillover effects.}
\end{frame}






\section{Conclusion}

\begin{frame}{Conclusion}
    \begin{itemize}
        \item I decomposed the TWFE estimate into the direct effect and two spillover terms
        
        \item I showed that a set of concentric rings removes two spillover terms from treatment effect estimate and models spillovers well
        
        \item For place-based policies, I show the importance of considering spatial spillovers when estimating treatment effects
    \end{itemize}
\end{frame}



\end{document}