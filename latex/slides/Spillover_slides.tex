\documentclass[aspectratio=169]{beamer}
% Metropolis Theme ------------------------------------------------------------------------------
\usetheme{metropolis} % Use metropolis theme


% Title ------------------------------------------------------------------------------
\title{Difference-in-Differences with Spatial Spillovers}
\date{\today}
\author{Kyle Butts}
% \institue{}

% Margins ----------------------------------------------------------------------

\usepackage[margin=1.25in]{geometry}

% AMS --------------------------------------------------------------------------

\usepackage{amsmath}
\usepackage{amsfonts}
\usepackage{amsthm}
\usepackage{graphicx}


% Line Spacing -----------------------------------------------------------------

\renewcommand{\baselinestretch}{1.5}


% Font -------------------------------------------------------------------------

\usepackage[T1]{fontenc}
\usepackage[default]{lato} % Lato as text font
% \usepackage[utopia, varg]{newtxmath}
% \renewcommand{\rmdefault}{futs} % Utopia as text font 

% Small adjustments to text kerning
\usepackage{microtype}

% Remove annoying over-full box warnings
\vfuzz2pt 
\hfuzz2pt


% Tikz support -----------------------------------------------------------------

\usepackage{tikz}


% Color Palette ----------------------------------------------------------------

\usepackage{xcolor}

% https://www.materialpalette.com/colors
\definecolor{dark-maroon}{HTML}{5D0F0D}
\definecolor{navyblue}{HTML}{0A3044}

% From Davidson Mackinnon
\definecolor{dm-blue}{HTML}{086fbd}
\definecolor{dm-red}{HTML}{ba3132}
\definecolor{dm-green}{HTML}{3f7e32}

% https://www.viget.com/articles/color-contrast/
\definecolor{purple}{HTML}{5601A4}
\definecolor{navy}{HTML}{0D3D56}
\definecolor{ruby}{HTML}{9a2515}
\definecolor{alice}{HTML}{107895}
\definecolor{daisy}{HTML}{EBC944}
\definecolor{coral}{HTML}{F26D21}
\definecolor{kelly}{HTML}{829356}
\definecolor{cranberry}{HTML}{E64173}
\definecolor{jet}{HTML}{131516}
\definecolor{asher}{HTML}{555F61}
\definecolor{slate}{HTML}{314F4F}


% Hyperlinks -------------------------------------------------------------------

\usepackage{hyperref}
\hypersetup{
    colorlinks= true,
    citecolor= dark-maroon,
    linkcolor= dark-maroon,
    filecolor= dark-maroon,      
    urlcolor= dark-maroon,
}


% Citations --------------------------------------------------------------------

% note, natbib provides better hyperlinking
\usepackage{natbib}
\bibliographystyle{econ-aea}


% Define Theorems --------------------------------------------------------------

% Put proper spacing after Theorem #. 
\newtheoremstyle{spacing}
{}%          Space above, empty = `usual value'
{}%          Space below
{}%  Body font
{}%          Indent amount (empty = no indent, \parindent = para indent)
{\bfseries\color{navyblue}}% Thm head font
{.}%         Punctuation after thm head
{2.5mm}%  Space after thm head: \newline = linebreak
{}%          Thm head spec

% note, theorem is the name that goes in \begin{} and Theorem is the name displayed as Theorem 1
\theoremstyle{spacing}
\newtheorem{theorem}{Theorem}
\newtheorem{proposition}{Proposition}
\newtheorem{assumption}{Assumption}
\newtheorem{example}{Example}


% Custom Math Definitions ------------------------------------------------------

\newcommand{\expec}[1]{\mathbb{E}\left[#1\right]}%
\newcommand{\condexpec}[2]{\mathbb{E}\left[#1 \ \vert \ #2\right]}%
\newcommand{\prob}[1]{\mathbb{P}\left[#1\right]}%
\newcommand{\var}[1]{\mathrm{Var}\left[#1\right]}%
\newcommand{\cov}[1]{\mathrm{Cov}\left[#1\right]}%
\newcommand{\one}{\mathbf{1}}


% Titlepage --------------------------------------------------------------------

% \maketitle
\usepackage{titling}
\usepackage{setspace}

% title
\pretitle{\begin{spacing}{1}\begin{flushleft}\huge}
\posttitle{\end{flushleft}\end{spacing}\vspace{-5mm}}
% author, note don't use \and 
\preauthor{\begin{flushleft}\LARGE}
\postauthor{\end{flushleft}\vspace{-7.5mm}}
% date
\predate{\begin{flushleft}\Large\color{asher}}
\postdate{\end{flushleft}\vspace{-5mm}}

% Abstract
\renewenvironment{abstract}
 {\noindent\rule{\linewidth}{.5pt}\noindent}
 {\noindent\rule{\linewidth}{.5pt}}

% alternative abstract
% \renewenvironment{abstract}
% {
%   \centerline {\large \bfseries \scshape \color{navyblue} Abstract}
%   \begin{quote}
% }
% {\end{quote}}


% Section and Subsection Styling -----------------------------------------------

\usepackage[explicit]{titlesec}

\titleformat{\section}
  {\Large \bf \color{navyblue}}
  {\thesection \,---}
  {0.25em}
  {#1}
  
\titleformat{\subsection}
  {\fontsize{11}{10}\it}
  {\thesubsection.}
  {1em}
  {#1}

% Don't number subsubsection
\setcounter{secnumdepth}{2}

% Footnote ---------------------------------------------------------------------

% Spacing between footnotes on same page
\addtolength{\footnotesep}{1mm}

% Space after footnote number
\let\oldfootnote\footnote
\renewcommand\footnote[1]{\oldfootnote{\ #1}}

% No footnote line
\renewcommand\footnoterule{}

% No supsercript in footer
\makeatletter
\renewcommand\@makefntext[1]{%
    \parindent 1em \noindent
    \hb@xt@1.8em{\hss\normalfont\@thefnmark.\hfill}#1
  }
\makeatother




% Enumerate/Itemize ------------------------------------------------------------

\usepackage{enumitem}
\setitemize{labelindent=0.5em,labelsep=0.25cm,leftmargin=*}
\setenumerate{labelindent=0.5em,labelsep=0.25cm,leftmargin=*}


% Table and Figure labelling ---------------------------------------------------

\usepackage{caption}

\DeclareCaptionLabelSeparator{threedash}{\,---\,}
\DeclareCaptionFont{navyblue}{\color{navyblue}}
\DeclareCaptionFont{jet}{\color{jet}}
\captionsetup[table]{format=plain, labelsep=threedash, font={navyblue, bf}}
\captionsetup[figure]{format=plain, labelsep=threedash, font={navyblue, bf}}

% Alternative: Left align captions
% \captionsetup[table]{labelfont=it, textfont={navyblue, bf}, labelsep=newline, justification=raggedright, singlelinecheck=off}
% \captionsetup[figure]{labelfont=it, textfont={navyblue, bf}, labelsep=newline, justification=raggedright, singlelinecheck=off}

% multifigure with \caption
% \begin{subfigure}\caption{} \end{subfigure}
\usepackage{subcaption}
\captionsetup[subfigure]{format=plain, font={jet, footnotesize, bf}}


% Tables -----------------------------------------------------------------------

% Fix \input with tables
% \input fails when \\ is at end of external .tex file

\makeatletter
\let\input\@@input
\makeatother

% Make tables/figures wider than \textwidth using:
% \begin{adjustbox}{width = 1.2\textwidth, center}
% \end{adjustbox}
\usepackage{adjustbox}

% Slighty more spacing between rows
\usepackage{array}
\renewcommand\arraystretch{1.2}

% Table with easy to use footnotes
% \begin{threeparttable}
%    \begin{tabular} ... \end{tabular}
%    \begin{tablenotes}
%        \item \textit{Notes.}
%    \end{tablenotes}  
% \end{threeparttable}
\usepackage[flushleft]{threeparttable}
\setlength\labelsep{0pt}

% \toprule, \cmidrule, \bottomrule
\usepackage{booktabs}

% If tables are too narrow, fill columns using:
% \begin{tabularx}{\linewidth}{cols}
% col-types: X - center, L - left, R -right
% If you want relative scale for columns: 
% >{\hsize=.8\hsize}X/L/R
\usepackage{tabularx}
\newcolumntype{L}{>{\raggedright\arraybackslash}X}
\newcolumntype{R}{>{\raggedleft\arraybackslash}X}
\newcolumntype{C}{>{\centering\arraybackslash}X}

% Shorter multicolumn commands
\newcommand{\mcc}[1]{\multicolumn{1}{c@{}}{#1}}
\newcommand{\mcl}[1]{\multicolumn{1}{l@{}}{#1}}
\newcommand{\mcr}[1]{\multicolumn{1}{r@{}}{#1}}

% d column
\usepackage{dcolumn}
\newcolumntype{d}[1]{D..{#1}}

% Landscape table 
% \begin{landscape} \pagestyle{lscaped} table... \end{landscsape}
% \usepackage{pdflscape} - rotates page left-side up in pdf
% \usepackage{lscape} - does not rotate page, only figure/table

\usepackage{pdflscape}

% For landscape, fix page number location
\usepackage{fancyhdr}
\fancypagestyle{lscaped}{%
    \fancyhf{}
    \renewcommand{\headrulewidth}{0pt}
    \textnormal
    \fancyfoot{%
        \tikz[remember picture,overlay]
        \node[outer sep=2.5cm,above,rotate=90] at (current page.east) {\thepage};
    }
}
  

% ------------------------------------------------------------------------------

\addbibresource{references.bib}

\usepackage{adjustbox}
\usepackage{tabularx}
\usepackage{booktabs}
\usepackage{threeparttable}
\usepackage{dcolumn} 

% Table Highlighting
\usepackage[beamer,customcolors]{hf-tikz}
\usetikzlibrary{calc}
\usetikzlibrary{fit,shapes.misc}

% To set the hypothesis highlighting boxes red.
\tikzset{hl/.style={
    set fill color=red!80!black!40,
    set border color=red!80!black,
  },
}
\newcommand\marktopleft[1]{%
    \tikz[overlay,remember picture] 
        \node (marker-#1-a) at (0,1.5ex) {};%
}
\newcommand\markbottomright[1]{%
    \tikz[overlay,remember picture] 
        \node (marker-#1-b) at (0,0) {};%
    \tikz[red, ultra thick, overlay, remember picture, inner sep=4pt]
        \node[draw, rectangle, fit=(marker-#1-a.center) (marker-#1-b.center)] {};%
}


% ------------------------------------------------------------------------------
\begin{document}

% ------------------------------------------------------------------------------
\maketitle
% ------------------------------------------------------------------------------

\begin{frame}{Spatial Spillovers}
    Researchers aim to estimate the \textbf{average treatment effect on the treated}: 
    \[
        \tau \equiv \mathbb{E} \left[ Y_{i1}(1) - Y_{i1}(0) \ \vert \ D_{i} = 1 \right]
    \]
    
    Estimation is complicated by \textbf{Spillover Effects}
    
    \vspace{5mm}
    \textbf{Spillover effects} are when effect of treatment extend over the treatment boundaries (states, counties, etc.). Example:
    
    \begin{itemize}
        \item A large employer opening/closing in a treated county have positive employment effects on \textbf{control counties}
        
        \item Having nearby counties with factories raises wages and reduces effect of \textbf{treated counties}
    \end{itemize}
\end{frame}

\begin{frame}{This Paper}
    In this paper, I

    \begin{itemize}
        \item Present a potential outcomes framework to formalize spillover effects and evaluate ad-hoc adjustments done in the literature
        
        \item Propose an estimation strategy that improves on current practices by being more robust to spillovers
        
        \item Apply this framework to improve estimation of the local effect of place-based policies in Urban Economics 
    \end{itemize}

\end{frame}


\begin{frame}{Bias from Spatial Spillovers}
    
    \onslide<1->{
        The canonical difference-in-differences estimate is: 
        \only<1>{
            \[ 
                \hat{\tau} = \underbrace{\hat{\mathbb{E}} \left[ Y_{i1} - Y_{i0} \mid D_i = 1 \right]}_{\text{Counterfactual Trend} \ + \ \tau} - 
                \underbrace{\hat{\mathbb{E}} \left[ Y_{i1} - Y_{i0} \mid D_i = 0 \right]}_{\text{Counterfactual Trend}}
            \]
        }
        \only<2>{
            \[ 
                \hat{\tau} = \underbrace{\hat{\mathbb{E}} \left[ Y_{i1} - Y_{i0} \mid D_i = 1 \right]}_{\text{Counterfactual Trend} \ + \ \tau} - 
                \underbrace{\hat{\mathbb{E}} \left[ Y_{i1} - Y_{i0} \mid D_i = 0 \right]}_{\substack{\text{Counterfactual Trend} \\[2mm] \ + \ \text{\color{purple} Spillover on Control}}}
            \]
        }
        \only<3>{
            \[ 
                \hat{\tau} = \underbrace{\hat{\mathbb{E}} \left[ Y_{i1} - Y_{i0} \mid D_i = 1 \right]}_{\substack{\text{Counterfactual Trend} \ + \ \tau \\[2mm] \ + \ \text{\color{red} Spillover on Treated}}} - 
                \underbrace{\hat{\mathbb{E}} \left[ Y_{i1} - Y_{i0} \mid D_i = 0 \right]}_{\substack{\text{Counterfactual Trend} \\[2mm] \ + \ \text{\color{purple} Spillover on Control}}}
            \]
        }
    } 

    Two problems in presence of spillover effects:
    
    
    \begin{itemize}
        \onslide<2->{
            \item {\bf \color{purple} Spillover onto Control Units:} 
            
            Nearby ``control'' units fail to estimate counterfactual trends because they are affected by treatment
        }
        
        \onslide<3->{
            \vspace{2.5mm}
            \item {\bf \color{red} Spillover onto other Treated Units:} 
            
            Treated units are also affected by nearby units and therefore combines ``direct'' effects with spillover effects
        }
    \end{itemize}

\end{frame}

\begin{frame}{Remove Bias}
    \only<1>{
        \[ y_{it} = \mu_t + \mu_i + \tau D_{it} + \varepsilon_{it} \]
    }
    \only<2>{
        \[ 
            y_{it} = \mu_t + \mu_i + \tau D_{it} + \text{Within}_{it} * D_{it} + \text{Within}_{it} * (1-D_{it}) + \varepsilon_{it}
        \]
    }

    \only<1>{ 
        \[ 
            \mathbb{E}\{\hat{\tau}\} = \tau + \text{\color{purple} Spillover on Control} + \text{\color{red}Spillover on Treated}
        \]
    }
    \only<2>{
        \[ 
            \mathbb{E}\{\hat{\tau}\} = \tau,
        \]
        so long as $\text{Within}_{it}$ contains all the units with spillovers.
    }
\end{frame}


% Grey out overlays
\setbeamercovered{transparent}

\begin{frame}{Outline}

    \begin{itemize}
        \item[1--] Formalize spillovers into a potential outcomes framework:
 
        \begin{citecolor}
            [\citet{Clarke_2017}, \citet{Berg_Streitz_2019}, and \citet{Verbitsky-Savitz_Raudenbush_2012}]
        \end{citecolor}

        \begin{itemize}
            \vspace{2.5mm}
            \item I decompose the difference-in-differences estimator into three parts: Direct Effect of Treatment, Spillover onto Treated Units, Spillover onto Control Units
            
            \vspace{2.5mm}
            \item Show that an indicator for being close to treated units remove \textit{all bias} so long as the indicator contains all units affected by spillovers
            
            \vspace{2.5mm}
            \item `Rings' are able to estimate spillover effects while still removing all bias
        \end{itemize} 

    \end{itemize}
\end{frame}

\begin{frame}{Outline}
    \begin{itemize}
        \item[2--] Apply framework to Urban Economics
        \begin{itemize}
            \item Revisit \begin{citecolor}\citet{Kline_Moretti_2014a}\end{citecolor} analysis of the Tennessee Valley Authority
            
            \begin{itemize}
                \vspace{2.5mm}
                \item The local effect estimate is contaminated by spillover effects to neighboring counties \begin{citecolor}\citep{Kline_Moretti_2014b}\end{citecolor}
                
                \vspace{2.5mm}
                \item Large scale manufacturing investment creates an `urban shadow' \begin{citecolor}\citep{Cuberes_Desmet_Rappaport_2021,Fujita_Krugman_Venables_2001}\end{citecolor}
            \end{itemize}
            
            \vspace{2.5mm}
            \item Discuss how framework can reconcile conflicting findings on effect of federal Empowerment Zones \begin{citecolor}\citep{Busso_Gregory_Kline_2013,Neumark_Kolko_2010}\end{citecolor}
        \end{itemize}
    \end{itemize} 
\end{frame}



% ------------------------------------------------------------------------------
\section{Theory}
% ------------------------------------------------------------------------------


\begin{frame}{Potential Outcomes Framework}
    For exposition, I will label units as counties. Assume all treatment occurs at the same time (2-periods or pre-post averages).\footnote{I extend this into an event study framework in the paper, but the intuition is the same as in the $2 \times 2$ setting.}
    
    \begin{itemize}
        \item $Y_{it}(D_i, \textcolor{blue}{h(\vec{D}, i)})$ is the potential outcome of county $i \in \{ 1, \dots, N \}$ at time $t$ with treatment status $D_i \in \{0, 1\}$.
        
        \item $\vec{D} \in \{0,1\}^N$ is the vector of all units treatments.
        
        \pause
        \item The function $\textcolor{blue}{h(\vec{D}, i)}$ maps the entire treatment vector into an `exposure mapping' which can be a scalar or a vector.
        
        \pause
        \item No exposure is when $\textcolor{blue}{h(\vec{D}, i)} = \vec{0}$.
    \end{itemize}
\end{frame}

% Grey out
% \setbeamercovered{transparent}

\begin{frame}{Examples of $h_i(\vec{D})$}
    
    Examples of $h_i(\vec{D})$:
    
    \begin{itemize}
        \item \textbf{Treatment within $x$ miles:}
        
        $\textcolor{blue}{h(\vec{D}, i)} = max_j \ 1(d(i, j) \leq x)$ where $d(i,j)$ is the distance between counties $i$ and $j$. 

        \begin{itemize}
            \item e.g. library access where $x$ is the maximum distance people will travel
            
            \item Spillovers are non-additive
        \end{itemize}

    \end{itemize}
\end{frame}

\imageframe{../../figures/figure-spill_within.png}

\begin{frame}{Examples of $h_i(\vec{D})$}
    
    Examples of $h_i(\vec{D})$:
    
    \begin{itemize}
        \item \textbf{Treatment within $x$ miles:}
        
        $\textcolor{blue}{h(\vec{D}, i)} = max_j \ 1(d(i, j) \leq x)$ where $d(i,j)$ is the distance between counties $i$ and $j$. 

        \begin{itemize}
            \item e.g. library access where $x$ is the maximum distance people will travel
            
            \item Spillovers are non-additive
        \end{itemize}
        
        \vspace{2.5mm}
        \item \textbf{Number of Treated within $x$ miles:}
        
        $\textcolor{blue}{h(\vec{D}, i)} = \sum_{j = 1}^k 1(d(i, j) \leq x)$. 

        \begin{itemize}
            \item e.g. large factories opening
            
            \item Agglomeration economies suggest spillovers are additive
        \end{itemize}

    \end{itemize}
\end{frame}


\imageframe{../../figures/figure-spill_within_additive.png}


\begin{frame}{Estimand of Interest}
    Estimand of Interest: 
    \[ 
        \textcolor{green}{\tau_{\text{direct}}} \equiv \textcolor{green}{\mathbb{E}\left[ Y_{i,1}(1, 0) - Y_{i,1}(0, 0) \mid D_i = 1\right]}
    \]


    \pause
    \vspace{2.5mm}
    Spillover Effects:

    \[
        \textcolor{red}{\tau_{\text{spillover, treated}}} \equiv \textcolor{red}{\mathbb{E} \left[ Y_{i1}(1, h_i(\vec{D})) - Y_{i1}(1, 0) \mid D_i = 1 \right]}
    \]

    \[ 
        \textcolor{purple}{\tau_{\text{spillover, control}}} = \textcolor{purple}{\mathbb{E} \left[ Y_{i1}(0, h_i(\vec{D})) - Y_{i1}(0, 0) \mid D_i = 0 \right]}
    \]
\end{frame}


\begin{frame}{Parallel Trends}
    I assume a modified version of the parallel counterfactual trends assumption: 

    \textbf{Assumption:} \textit{Parallel Counterfactual Trends}
    \begin{align*}
        &\mathbb{E}\big[ \underbrace{Y_{i,1}(0, \textcolor{blue}{\vec{0}}) - Y_{i,0}(0, \textcolor{blue}{\vec{0}})}_{\text{Counterfactual Trend}} \mid D_i = 1 \big] \\
        = \ &\mathbb{E}\big[ \underbrace{Y_{i,1}(0, \textcolor{blue}{\vec{0}}) - Y_{i,0}(0, \textcolor{blue}{\vec{0}})}_{\text{Counterfactual Trend}} \mid D_i = 0 \big],
    \end{align*}

    \vspace{5mm}
    In the \textit{complete absence of treatment }(not just the absence of individual $i$'s treatment):
    
    Changes in outcomes does not depend on treatment status
    
\end{frame}

\begin{frame}{What does Difference-in-Differences identify?}
    With the parallel trends assumption and random assignment of $D_i$, I decompose the difference-in-differences estimate as follows: 
        
    \begin{align*}
        \mathbb{E}\left[\hat{\tau}\right] &= \underbrace{\mathbb{E} \left[ Y_{i1} - Y_{i0} \mid D_i = 1 \right] - \mathbb{E} \left[ Y_{i1} - Y_{i0} \mid D_i = 0 \right]}_{\text{Difference-in-Differences}} \\[3mm]
        \pause&= 
        \textcolor{green}{\mathbb{E} \left[ Y_{i1}(1, 0) - Y_{i1}(0, 0) \mid D_i = 1 \right]} \\
        &\quad + \quad 
        \textcolor{red}{\mathbb{E} \left[ Y_{i1}(1, h_i(\vec{D})) - Y_{i1}(1, 0) \mid D_i = 1 \right]} \\ 
        &\quad - \quad  
        \textcolor{purple}{\mathbb{E} \left[ Y_{i1}(0, h_i(\vec{D})) - Y_{i1}(0, 0) \mid D_i = 0 \right]} \\[3mm]
        &= \textcolor{green}{\tau_{\text{direct}}} + \textcolor{red}{\tau_{\text{spillover, treated}}} - \textcolor{purple}{\tau_{\text{spillover, control}}}
    \end{align*}

\end{frame}


\begin{frame}{Removing Bias}
    \textbf{Assumption:} \textit{Spillovers are Local}

    Let $d(i,j)$ be the distance between units $i$ and $j$. There exists a distance $\bar{d}$ such that 
        
        (i) For all units $i$,
        \[ 
            \min_{j: \ D_j = 1} d(i,j) > \bar{d} \implies h(\vec{D}, i) = \vec{0}. 
        \]
    
        (ii) There are treated and control units such that $\min_{j: \ D_j = 1} d(i,j) > \bar{d}$.
\end{frame}

\begin{frame}{Removing Bias}
    With assumption that spillovers are local, define $S_i$ to be an indicator that contains all units with $h(\vec{D}, i) \neq \vec{0}$ (and potentially some units with $= \vec{0}$).
    
    Then $\hat{\tau}$ will be a consistent estimate for $\tau$ in this specification: 
    \[ 
        y_{it} = \mu_t + \mu_i + \tau D_{it} + \text{Within}_{it} * D_{it} + \text{Within}_{it} * (1-D_{it}) + \varepsilon_{it}
    \]
    

\end{frame}



% ------------------------------------------------------------------------------
\section{Estimation with Spillovers}
% ------------------------------------------------------------------------------


\begin{frame}{Common Solution: Removing ``contaminated'' controls}
    A common solution to the problem of spillover is to restimate on a subsample with neighboring control units removed.

    This is not recommended for two reasons:

    \begin{itemize}
        \item Removing control units from the sample decreases precision of the estimates
        
        \item Spillover effects on treated units will still remain in the estimand. 
    \end{itemize}
\end{frame}

\begin{frame}{Solution: Parametrize potential outcomes}
    Another option is to directly control for spillovers by parametrizing the exposure mapping. 
    
    The corret exposure mapping is not observable to a researcher which raises the question:

    Are there specifications that work well even if the spillovers are misspecified?
\end{frame}

\begin{frame}{Robustness to Misspecification}
    Generate data using the same data-generating process as before but with different spillover functions:

    \[ 
        y_{\it} = \mu_t + \mu_i + 2 D_{it} + \beta_{\text{spill,control}} * (1-D_{it}) h(\vec{D}, i) + \varepsilon_{it}
    \]

    \begin{itemize}
        \item Observation $i$ is a U.S. county, year $t \in \{2000, \dots, 2019\}$, treatment $D_{it}$ turns on in 2010 and is assigned randomly.
    \end{itemize}

    Then, I estimate each data-generating process using (potentially) misspecified $\tilde{h}(\vec{D}, i)$ and report the average estimate bias.
\end{frame}

\begin{frame}{Specifications of $h(\vec{D}, i)$}
    
    \begin{itemize}
        \item \textit{Within $40/80$mi.}:
        \begin{itemize}
            \item Indicator for nearest treated unit being within $40/80$ miles.
        \end{itemize}
        
        \item \textit{Within $40/80$mi. (Additive)}: 
        \begin{itemize}
            \item Number of treated units being within $40/80$ miles.
        \end{itemize}
        
    \end{itemize}

\end{frame}

\imageframe{../../figures/figure-spill_within_large.png}
\imageframe{../../figures/figure-spill_within_large_additive.png}

\begin{frame}{Specifications of $h(\vec{D}, i)$}
    
    \begin{itemize}
        \item \textit{Within $40/80$mi.}:
        \begin{itemize}
            \item Indicator for nearest treated unit being within $40/80$ miles.
        \end{itemize}
        
        \item \textit{Within $40/80$mi. (Additive)}: 
        \begin{itemize}
            \item Number of treated units being within $40/80$ miles.
        \end{itemize}
        

        \item \textit{Decay}: 
        \begin{itemize}
            \item $\max_j D_j * e^{-0.02 d(i,j)} * 1(d(i,j) < 80)$
        \end{itemize}
        
        \item \textit{Decay (Additive)}: 
        \begin{itemize}
            \item $\sum_j D_j * e^{-0.02 d(i,j)} * 1(d(i,j) < 80)$
        \end{itemize}
        
    \end{itemize}

\end{frame}


\imageframe{../../figures/figure-spill_decay.png}
\imageframe{../../figures/figure-spill_decay_additive.png}


\begin{frame}{Specifications of $h(\vec{D}, i)$}
    
    \begin{itemize}
        \item \textit{Within $40/80$mi.}:
        \begin{itemize}
            \item Indicator for nearest treated unit being within $40/80$ miles.
        \end{itemize}
        
        \item \textit{Within $40/80$mi. (Additive)}: 
        \begin{itemize}
            \item Number of treated units being within $40/80$ miles.
        \end{itemize}
        
        \item \textit{Decay}: 
        \begin{itemize}
            \item $\max_j D_j * e^{-0.02 d(i,j)} * 1(d(i,j) < 80)$
        \end{itemize}
        
        \item \textit{Decay (Additive)}: 
        \begin{itemize}
            \item $\sum_j D_j * e^{-0.02 d(i,j)} * 1(d(i,j) < 80)$
        \end{itemize}
        
        \item \textit{Rings}: 
        \begin{itemize}
            \item Set of concentric rings. For each ring, indicator for nearest treated unit being within that distance bin
        \end{itemize}
        
        \item \textit{Rings (Additive)}: 
        \begin{itemize}
            \item Set of concentric rings. For each ring, number of treated units being within that distance bin
        \end{itemize}
    \end{itemize}

\end{frame}

\imageframe{../../figures/figure-spill_ring.png}

\begin{frame}

    \begin{table}[!tb]
        \caption{Bias from Misspecification of Spillovers}
        \label{tab:misspecification}
    
        \begin{adjustbox}{width = 1.1\textwidth, center}
            \begin{threeparttable}
                \begin{tabular}{@{} l rrrrrr @{}}
                    % Head
                    \toprule
                    & \multicolumn{6}{c}{Data-Generating Process} \\
                    \cmidrule{2-7}
    
                    & \multicolumn{1}{c}{\textbf{Within 40mi.}} & \multicolumn{1}{c}{\textbf{Within 80mi.}} & \multicolumn{1}{c}{\textbf{Within 40mi.}} & \multicolumn{1}{c}{\textbf{Within 80mi.}} & \multicolumn{1}{c}{\textbf{Decay 80mi.}} & \multicolumn{1}{c}{\textbf{Decay 80mi.}} \\
                    Specification & & & \multicolumn{1}{c}{\textbf{(Additive)}} & \multicolumn{1}{c}{\textbf{(Additive)}} & & \multicolumn{1}{c}{\textbf{(Additive)}} \\
    
     
                    % Body
                    \midrule
                    
                    
                    % \input{../../tables/misspecification_slides.tex}
                    TWFE (No Spillovers) & $0.258$ & $0.258$ & $0.258$ & $0.258$ & $0.258$ & $0.258$ \\ \midrule
                    \marktopleft{c1}Within 40mi. & $-0.005$ & $0.213$ & $-0.005$ & $0.176$ & $0.159$ & $0.143$ \\ 
                    Within 80mi. & $-0.009$ & $-0.009$ & $-0.009$ & $-0.009$ & $-0.009$ & $-0.009$ \markbottomright{c1}
                    \onslide<2->{
                        \\ \midrule
                        Within 40mi. (Additive) & $0.043$ & $0.221$ & $-0.006$ & $0.177$ & $0.174$ & $0.143$ \\ 
                        Within 80mi. (Additive) & $0.034$ & $0.134$ & $-0.012$ & $-0.009$ & $0.099$ & $-0.010$ \\ \midrule
                        Decay 80mi. & $-0.159$ & $0.070$ & $-0.174$ & $0.014$ & $-0.009$ & $-0.033$ \\ 
                        Decay 80mi. (Additive) & $-0.023$ & $0.148$ & $-0.084$ & $0.019$ & $0.088$ & $-0.008$
                    }
                    \onslide<3->{
                        \\ \midrule
                        \marktopleft{c2}Rings (0-20, 20-30, 30-40) & $-0.005$ & $0.213$ & $-0.005$ & $0.176$ & $0.159$ & $0.143$ \\ 
                        Rings (0-20, 20-30, 30-40, 40-60, 60-80) & $-0.009$ & $-0.009$ & $-0.009$ & $-0.009$ & $-0.009$ & $-0.009$ \markbottomright{c2}\\ 
                        Rings (0-20, 20-30, 30-40, 40-60, 60-80) (Additive) & $0.036$ & $0.134$ & $-0.008$ & $-0.008$ & $0.100$ & $-0.009$ 
                    }
                        
                    \\ \bottomrule
                \end{tabular}
            \end{threeparttable}
        \end{adjustbox}
    \end{table}

    \onslide<3->{
        \textbf{Indicator (or set of indicators) that captures all affected unit removes all bias}
    }
\end{frame}

\begin{frame}{Spillovers as Estimand of Interest}
    Until now, we assumed our estimand of interest is $\textcolor{green}{\tau_{\text{direct}}}$.
    
    However, the two other spillover effects are of interest as well:
    \begin{itemize}
        \item $\textcolor{purple}{\tau_{\text{spillover, control}}}$: Do the benefits of a treated county come at a cost to neighbor counties? 
        
        \item $\textcolor{red}{\tau_{\text{spillover, treated}}}$: Does the estimated effect change based on treatment of neighbors? 
    \end{itemize}
    
    To estimate the spillover effects, we have to parameterize $h(\vec{D}, i)$ function and the potential outcomes function $Y_i(D_i, h(\vec{D}, i))$.
\end{frame}

\begin{frame}{Estimation of Spillover Effects}
    To see which specifications can predict spillover effects well, I estimate the spillover effects for each control unit, $\hat{\beta}_{\text{spill, control}} * \tilde{h}(\vec{D}, i)$.

    Then calculate \[ 
        1 - \frac{
                \overbrace{\sum_{i: D_i = 0} (\beta_{\text{spill, control}} h(\vec{D}, i) - \hat{\beta}_{\text{spill, control}} \tilde{h}(\vec{D}, i))^2}^{\text{Mean Square Prediction Error}}
            }{
                \underbrace{\sum_{i: D_i = 0} (\beta_{\text{spill, control}} h(\vec{D}, i))^2}_{\text{Normalization}}
            }    
    \]

    This gives the proportion of spillovers explained by $\tilde{h}(\vec{D}, i)$

\end{frame}


\begin{frame}
    \begin{table}[!tb]
        \caption{Percent of Spillovers Predicted by Specification}
        \label{tab:misspecification_mspe}
    
        \begin{adjustbox}{width = 1.1\textwidth, center}
            \begin{threeparttable}
                \begin{tabular}{@{} l rrrrrr @{}}
                    % Head
                    \toprule
                    & \multicolumn{6}{c}{Data-Generating Process} \\
                    \cmidrule{2-7}
    
                    & \multicolumn{1}{c}{\textbf{Within 40mi.}} & \multicolumn{1}{c}{\textbf{Within 80mi.}} & \multicolumn{1}{c}{\textbf{Within 40mi.}} & \multicolumn{1}{c}{\textbf{Within 80mi.}} & \multicolumn{1}{c}{\textbf{Decay 80mi.}} & \multicolumn{1}{c}{\textbf{Decay 80mi.}} \\
                    Specification & & & \multicolumn{1}{c}{\textbf{(Additive)}} & \multicolumn{1}{c}{\textbf{(Additive)}} & & \multicolumn{1}{c}{\textbf{(Additive)}} \\
    
     
                    % Body
                    \midrule
                    
                    
                    % \input{../../tables/misspecification_slides.tex}
                    TWFE (No Spillovers) & $0.0\%$ & $0.0\%$ & $0.0\%$ & $0.0\%$ & $0.0\%$ & $0.0\%$ \\ \midrule
                    \marktopleft{c3}Within 40mi. & $99.4\%$ & $25.9\%$ & $85.6\%$ & $38.8\%$ & $59.5\%$ & $56.1\%$ \\ 
                    Within 80mi. & $39.8\%$ & $96.2\%$ & $34.3\%$ & $71.7\%$ & $85.6\%$ & $68.0\%$ \markbottomright{c3}
                    \onslide<2->{
                    \\ \midrule
                        Within 40mi. (Additive) & $85.3\%$ & $21.2\%$ & $99.5\%$ & $40.6\%$ & $52.0\%$ & $60.7\%$ \\ 
                        Within 80mi. (Additive) & $45.8\%$ & $61.8\%$ & $47.2\%$ & $98.4\%$ & $71.0\%$ & $93.6\%$ \\ \midrule 
                        Decay 80mi. & $60.1\%$ & $82.5\%$ & $52.7\%$ & $75.8\%$ & $97.5\%$ & $82.2\%$ \\ 
                        Decay 80mi. (Additive) & $60.7\%$ & $56.9\%$ & $63.8\%$ & $93.5\%$ & $79.0\%$ & $98.7\%$ 
                    }
                    \onslide<3->{
                        \\ \midrule 
                        \marktopleft{c4}Rings (0-20, 20-30, 30-40) & $98.4\%$ & $23.7\%$ & $85.9\%$ & $37.5\%$ & $58.9\%$ & $56.2\%$ \\ 
                        Rings (0-20, 20-30, 30-40, 40-60, 60-80) & $96.6\%$ & $91.7\%$ & $84.2\%$ & $72.7\%$ & $91.9\%$ & $78.4\%$ \\ 
                        Rings (0-20, 20-30, 30-40, 40-60, 60-80) (Additive) & $83.5\%$ & $57.4\%$ & $97.6\%$ & $95.0\%$ & $73.5\%$ & $94.9\%$\markbottomright{c4} 
                    }
                        
                    
                    \\ \bottomrule
                \end{tabular}
            \end{threeparttable}
        \end{adjustbox}
    \end{table}

    \onslide<3->{
        \textbf{Donuts perform best at estimating spillover effects.}
    
        \textbf{It is important to get Additive vs. Non-Additive correct.}
    }
\end{frame}




% ------------------------------------------------------------------------------
\section{Application in Urban Economics}
% ------------------------------------------------------------------------------

\begin{frame}{Tennessee Valley Authority}
    \citet{Kline_Moretti_2014a} look at the long-run impacts of the Tennessee Valley Authority (TVA).

    \begin{itemize}
        \item The TVA was a large-scale federal investment started in 1934 that focused on improving manufacturing economy. (Hundreds of dollars spent anually per person)
        
        \item The program focused on large-scale dam construction that brought cheap wholesale electricity to the region
    \end{itemize}

    Research Question: What are the local effects of TVA on manufacturing and agricultural economies? Do these effects come at the cost of other counties?
\end{frame}

\begin{frame}{Identification}
    \citet{Kline_Moretti_2014a} run the county-level difference-in-differences  specification: 
    \begin{equation}\label{eq:tva_spillover}
        y_{c, 2000} - y_{c, 1940} = \alpha + \text{TVA}_c \tau + X_{c, 1940} \beta + (\varepsilon_{c, 2000} - \varepsilon_{c, 1940})
    \end{equation} 

    \begin{itemize}
        \item $y$ are outcomes for agricultural employment and manufacturing employment.
        \item $\text{TVA}_c$ is the treatment variable
        \item $X_{c, 1940}$ allow for different long-term trends based on covariates in 1940. 
    \end{itemize}

    They trim the sample using a logit regression to predict treatment using $X_{c,1940}$ and then keep control units in the top 75\% of predicted probability.
\end{frame}

\begin{frame}{Spillovers in the TVA Context}
    In our context, there is reason to believe spillovers can occur to nearby counties

    \begin{itemize}
        \item \textbf{Agriculture:} 
        \begin{itemize}
            \item Employees might be drawn to hire wages for new manufacturing jobs in Tennessee Valley (negative spillover on control units)
        \end{itemize}
        
        \item \textbf{Manufacturing:}
        \begin{itemize}
            \item Cheap electricity might be available to nearby counties (positive spillover on control units) 
            
            \item Manufacturing jobs that would have been created in the control units in the absence of treatment might move to the Tennessee Valley (negative spillover on control units)
        \end{itemize}
    \end{itemize}
\end{frame}

\begin{frame}{Specification including spillovers}
    \begin{equation}
        \Delta y_c = \alpha + \text{TVA}_i \tau + \sum_{d \in \text{Dist}} \text{Ring}(d)\delta_d + X_{i, 1940} \beta + \Delta \varepsilon_c
    \end{equation} 

    \begin{itemize}
        \item $\text{Ring}(d)$ is a set of indicators for being in the following distance bins (in miles) from the Tennessee Valley Authority: 
        \[ d \in \left\{ (0, 50], (50, 100], (100, 150] \right\} \]
    \end{itemize}
\end{frame}

\imageframe{../../figures/figure-tva-sample_slides.pdf}

\begin{frame}
\begin{table}[ht]
    \caption{Effects of Tennessee Valley Authority on Decadel Growth, 1940-2000}
    \label{tab:tva}
    \renewcommand{\arraystretch}{1.2}

    \begin{adjustbox}{width = 1\textwidth, center}
        \begin{threeparttable}
            \begin{tabular}{@{} lc@{\extracolsep{20pt}}c@{\extracolsep{4pt}}ccc @{}}
                % Head
                \toprule

                & \multicolumn{1}{c}{\textbf{Diff-in-Diff}} & \multicolumn{4}{c}{\textbf{Diff-in-Diff with Spillovers}} \\ 
                \cmidrule{2-2} \cmidrule{3-6} 
                & & & TVA between & TVA between & TVA between \\ 
                \textit{Dependent Var.} & TVA & TVA & 0-50 mi. & 50-100 mi. & 100-150 mi. \\ 

                % Body
                \midrule
                
                Agricultural employment     & $-0.0514^{***}$& $-0.0678^{***}$& $-0.0310^{**}$ &    $-0.0112$   & $-0.0252^{***}$\\
                &   $(0.0087)$   &   $(0.0102)$   &   $(0.0123)$   &   $(0.0094)$   &   $(0.0084)$ 

                \onslide<2->{
                    \\
                    Manufacturing employment    & $0.0560^{***}$ &  $0.0461^{**}$ &    $-0.0104$   &    $-0.0128$   &  $-0.0248^{*}$ \\
                    &   $(0.0187)$   &   $(0.0210)$   &   $(0.0205)$   &   $(0.0257)$   &   $(0.0147)$
                }
               
                \\ \bottomrule
            \end{tabular}
            
            % Notes 
            \begin{tablenotes}\footnotesize

                \item $^{*} p< 0.1$; $^{**} p < 0.05$; $^{***} p < 0.01$.
            \end{tablenotes}
        \end{threeparttable}
    \end{adjustbox}
\end{table}    

\end{frame}


\begin{frame}{Identification Strategies and Place-Based Policies}

    The literature on federal Enterprise Zones, place-based policy that gives tax breaks to businesses that locate within the boundary, has found conflicting results, suggesting positive or near-zero effects of the program \citep{Neumark_Young_2019}. 
    
    \begin{itemize}
        \pause
        \item \citet{Busso_Gregory_Kline_2013} compare census tracts in Empowerment Zones to census tracts that qualified and were rejected from the program. They find significant large reduction of poverty.
        
        \pause
        \item \citet{Neumark_Kolko_2010} compare census tracts in Empowerment Zones to census tracts within 1,000 feet of the Zone. They find near-zero effects on poverty.
    \end{itemize}

    \pause
    My framework can explain both of these results. If census tracts just outside the Empowerment Zones also benefit from the policy, then the estimates of \citet{Neumark_Kolko_2010} are attenuated towards zero

\end{frame}





\section{Conclusion}

\begin{frame}{Conclusion}
    \begin{itemize}
        \item I decomposed the TWFE estimate into the direct effect and two spillover terms
        
        \item I showed that a set of concentric rings allows for estimation of the direct effect of treatment and they are able to model spillovers well
        
        \item For place-based policies, I show the importance of considering spatial spillovers when estimating treatment effects
        
        \item More generally, identification strategies that use very close control units in order to minimize differences in unobservables should consider the problems with treatment effect spillovers.
    \end{itemize}
\end{frame}





















\end{document}
